% Options for packages loaded elsewhere
\PassOptionsToPackage{unicode}{hyperref}
\PassOptionsToPackage{hyphens}{url}
\PassOptionsToPackage{dvipsnames,svgnames,x11names}{xcolor}
%
\documentclass[
  11.5pt,
]{article}

\usepackage{amsmath,amssymb}
\usepackage{iftex}
\ifPDFTeX
  \usepackage[T1]{fontenc}
  \usepackage[utf8]{inputenc}
  \usepackage{textcomp} % provide euro and other symbols
\else % if luatex or xetex
  \usepackage{unicode-math}
  \defaultfontfeatures{Scale=MatchLowercase}
  \defaultfontfeatures[\rmfamily]{Ligatures=TeX,Scale=1}
\fi
\usepackage{lmodern}
\ifPDFTeX\else  
    % xetex/luatex font selection
\fi
% Use upquote if available, for straight quotes in verbatim environments
\IfFileExists{upquote.sty}{\usepackage{upquote}}{}
\IfFileExists{microtype.sty}{% use microtype if available
  \usepackage[]{microtype}
  \UseMicrotypeSet[protrusion]{basicmath} % disable protrusion for tt fonts
}{}
\makeatletter
\@ifundefined{KOMAClassName}{% if non-KOMA class
  \IfFileExists{parskip.sty}{%
    \usepackage{parskip}
  }{% else
    \setlength{\parindent}{0pt}
    \setlength{\parskip}{6pt plus 2pt minus 1pt}}
}{% if KOMA class
  \KOMAoptions{parskip=half}}
\makeatother
\usepackage{xcolor}
\usepackage[top=0.5in, bottom=1in, paperwidth= 8.5in, paperheight=
11in,]{geometry}
\setlength{\emergencystretch}{3em} % prevent overfull lines
\setcounter{secnumdepth}{-\maxdimen} % remove section numbering
% Make \paragraph and \subparagraph free-standing
\ifx\paragraph\undefined\else
  \let\oldparagraph\paragraph
  \renewcommand{\paragraph}[1]{\oldparagraph{#1}\mbox{}}
\fi
\ifx\subparagraph\undefined\else
  \let\oldsubparagraph\subparagraph
  \renewcommand{\subparagraph}[1]{\oldsubparagraph{#1}\mbox{}}
\fi


\providecommand{\tightlist}{%
  \setlength{\itemsep}{0pt}\setlength{\parskip}{0pt}}\usepackage{longtable,booktabs,array}
\usepackage{calc} % for calculating minipage widths
% Correct order of tables after \paragraph or \subparagraph
\usepackage{etoolbox}
\makeatletter
\patchcmd\longtable{\par}{\if@noskipsec\mbox{}\fi\par}{}{}
\makeatother
% Allow footnotes in longtable head/foot
\IfFileExists{footnotehyper.sty}{\usepackage{footnotehyper}}{\usepackage{footnote}}
\makesavenoteenv{longtable}
\usepackage{graphicx}
\makeatletter
\def\maxwidth{\ifdim\Gin@nat@width>\linewidth\linewidth\else\Gin@nat@width\fi}
\def\maxheight{\ifdim\Gin@nat@height>\textheight\textheight\else\Gin@nat@height\fi}
\makeatother
% Scale images if necessary, so that they will not overflow the page
% margins by default, and it is still possible to overwrite the defaults
% using explicit options in \includegraphics[width, height, ...]{}
\setkeys{Gin}{width=\maxwidth,height=\maxheight,keepaspectratio}
% Set default figure placement to htbp
\makeatletter
\def\fps@figure{htbp}
\makeatother
% definitions for citeproc citations
\NewDocumentCommand\citeproctext{}{}
\NewDocumentCommand\citeproc{mm}{%
  \begingroup\def\citeproctext{#2}\cite{#1}\endgroup}
\makeatletter
 % allow citations to break across lines
 \let\@cite@ofmt\@firstofone
 % avoid brackets around text for \cite:
 \def\@biblabel#1{}
 \def\@cite#1#2{{#1\if@tempswa , #2\fi}}
\makeatother
\newlength{\cslhangindent}
\setlength{\cslhangindent}{1.5em}
\newlength{\csllabelwidth}
\setlength{\csllabelwidth}{3em}
\newenvironment{CSLReferences}[2] % #1 hanging-indent, #2 entry-spacing
 {\begin{list}{}{%
  \setlength{\itemindent}{0pt}
  \setlength{\leftmargin}{0pt}
  \setlength{\parsep}{0pt}
  % turn on hanging indent if param 1 is 1
  \ifodd #1
   \setlength{\leftmargin}{\cslhangindent}
   \setlength{\itemindent}{-1\cslhangindent}
  \fi
  % set entry spacing
  \setlength{\itemsep}{#2\baselineskip}}}
 {\end{list}}
\usepackage{calc}
\newcommand{\CSLBlock}[1]{\hfill\break\parbox[t]{\linewidth}{\strut\ignorespaces#1\strut}}
\newcommand{\CSLLeftMargin}[1]{\parbox[t]{\csllabelwidth}{\strut#1\strut}}
\newcommand{\CSLRightInline}[1]{\parbox[t]{\linewidth - \csllabelwidth}{\strut#1\strut}}
\newcommand{\CSLIndent}[1]{\hspace{\cslhangindent}#1}

\usepackage{caption}
\usepackage{subcaption}
\usepackage{amsmath}
\usepackage{float}
\usepackage{tablefootnote}
\floatplacement{figure}{H}
\usepackage{tikz}
\usepackage{pdflscape}
\makeatletter
\@ifpackageloaded{caption}{}{\usepackage{caption}}
\AtBeginDocument{%
\ifdefined\contentsname
  \renewcommand*\contentsname{Table of contents}
\else
  \newcommand\contentsname{Table of contents}
\fi
\ifdefined\listfigurename
  \renewcommand*\listfigurename{List of Figures}
\else
  \newcommand\listfigurename{List of Figures}
\fi
\ifdefined\listtablename
  \renewcommand*\listtablename{List of Tables}
\else
  \newcommand\listtablename{List of Tables}
\fi
\ifdefined\figurename
  \renewcommand*\figurename{Figure}
\else
  \newcommand\figurename{Figure}
\fi
\ifdefined\tablename
  \renewcommand*\tablename{Table}
\else
  \newcommand\tablename{Table}
\fi
}
\@ifpackageloaded{float}{}{\usepackage{float}}
\floatstyle{ruled}
\@ifundefined{c@chapter}{\newfloat{codelisting}{h}{lop}}{\newfloat{codelisting}{h}{lop}[chapter]}
\floatname{codelisting}{Listing}
\newcommand*\listoflistings{\listof{codelisting}{List of Listings}}
\makeatother
\makeatletter
\makeatother
\makeatletter
\@ifpackageloaded{caption}{}{\usepackage{caption}}
\@ifpackageloaded{subcaption}{}{\usepackage{subcaption}}
\makeatother
\ifLuaTeX
  \usepackage{selnolig}  % disable illegal ligatures
\fi
\usepackage{bookmark}

\IfFileExists{xurl.sty}{\usepackage{xurl}}{} % add URL line breaks if available
\urlstyle{same} % disable monospaced font for URLs
\hypersetup{
  pdftitle={Internalizing Environmental Risk: Insurance Design and Firm Behavior in Hazardous Industries},
  pdfauthor={Kaleb K. Javier},
  colorlinks=true,
  linkcolor={blue},
  filecolor={Maroon},
  citecolor={Blue},
  urlcolor={Blue},
  pdfcreator={LaTeX via pandoc}}

\title{Internalizing Environmental Risk: Insurance Design and Firm
Behavior in Hazardous Industries}
\author{Kaleb K. Javier}
\date{Invalid Date}

\begin{document}
\maketitle
\begin{abstract}
Public trust funds in thirty-eight U.S. states insure underground
storage tank (UST) leaks at a uniform flat fee, even though leak risk
varies widely with observable traits such as tank age and wall
construction. This paper quantifies how replacing flat-fee pooling with
actuarially priced coverage changes firm behavior and environmental
outcomes. I exploit Texas's 1999 closure of its trust fund, which forced
all owners into a private market that sets premiums according to
facility risk. A facility-year panel covering 1990--2021 links EPA
inventories with administrative microdata for Texas and eighteen control
states that retained public funds. Difference-in-differences and
event-study estimates show that, among single-wall tanks installed
before 1999, risk-rated premiums increased annual closure rates by two
to three percentage points, shifted closures from market exit to on-site
replacement with safer equipment, and reduced the probability of a
reported leak by about one percentage point, nearly the entire
pre-policy baseline for this high-risk group. The findings provide the
first causal micro-evidence that cost-based environmental liability
insurance can internalize pollution risk more effectively than uniform
public pooling, offering clear guidance for states that are
reconsidering the structure of UST insurance programs.
\end{abstract}

\clearpage

\section{Introduction}\label{introduction}

Underground storage tanks holding motor fuels and other hazardous
liquids sit beneath about half a million U.S. facilities, and leaks from
these tanks are the leading documented source of groundwater
contamination EPA (\citeproc{ref-epa_underground_2024}{2024}). Cleanup
is costly; median remediation expenses exceed \$100 000 per site Marcus
(\citeproc{ref-marcus_going_2021}{2021}), and the probability of a
release rises sharply with observable characteristics such as tank age
and whether the walls are single or double. Despite these predictable
differences, thirty-eight states finance remediation through public
trust funds that charge every facility the same premium, diluting
incentives to invest in safer technology and crowding out private
insurers ASTSWMO (\citeproc{ref-astswmo_archive}{2025}).

Economic theory predicts that tying premiums to expected damages should
induce firms to internalize environmental risk Shavell
(\citeproc{ref-shavell_liability_1984}{1984b}); Polinsky
(\citeproc{ref-polinsky_strict_1980}{1980}), yet empirical tests of this
mechanism for USTs are scarce. Michigan's mid-1990s shift toward
private, risk-rated coverage suggested large gains but relied on
aggregate counts that could not isolate facility-level responses Yin,
Pfaff, and Kunreuther (\citeproc{ref-yin_can_2011}{2011}); Yin,
Kunreuther, and White (\citeproc{ref-yin_risk-based_2011}{2011}).
Whether actuarial pricing disciplines high-risk owners without simply
driving them out of the market therefore remains unresolved.

Texas provides a clean natural experiment. In 1999 the state closed its
public fund overnight and required all owners to purchase private
coverage whose premiums vary with observable risk. I assemble a new
facility-year panel covering 1990 to 2021 that links EPA inventories
with administrative microdata from the Texas Commission on Environmental
Quality and comparable data from eighteen control states that retained
trust funds. A difference-in-differences design compares Texas with
these fund states, and an event-study variant confirms parallel
pre-trends.

Three main results emerge. First, among single-wall tanks installed
before 1999, risk-based pricing raised annual closure probabilities by
roughly two to three percentage points and shifted closures from
outright exit to on-site replacement with safer equipment. Second, the
probability of a reported leak fell by about one percentage point,
nearly the entire pre-policy baseline for this high-risk group. Third,
the effect on leak rates is concentrated among facilities that were
already high-risk before the policy change, suggesting that cost-based
pricing effectively targets riskier firms. By providing the first causal
micro-evidence that actuarially priced environmental liability insurance
changes firm behavior in hazardous industries, the paper speaks directly
to current policy debates over the future of state UST funds.

\textbf{Related Literature}

A well-established body of theoretical work suggests that strict
liability rules can be an effective policy instrument for environmental
risks, particularly when paired with insurance contracts that align a
firm's private incentives with social costs (Shavell
(\citeproc{ref-shavell_liability_1984}{1984b}), Shavell
(\citeproc{ref-shavell_model_1984}{1984a}), Shavell
(\citeproc{ref-shavell_liability_1982}{1982}), Polinsky
(\citeproc{ref-polinsky_strict_1980}{1980}), Boyd and Kunreuther
(\citeproc{ref-boyd_retroactive_1997}{1997}), Shavell
(\citeproc{ref-shavell_chapter_2007}{2007})). A crucial condition for
this alignment is the insurer's ability to accurately price the risk of
environmental harm. This paper provides the first empirical test of this
principle in the context of environmental liability, offering causal
estimates of how firms respond to a transition from flat-fee to more
efficient, cost-based insurance contracts.

The challenge of pricing environmental risk connects this study to the
broader economics literature on insurance and selection markets.
Specifically, this paper builds on work examining the effects of
cost-based pricing and contract design on market outcomes and
participant behavior (Janvry, McIntosh, and Sadoulet
(\citeproc{ref-de_janvry_supply-_2010}{2010}), McWilliams, Hsu, and
Newhouse (\citeproc{ref-mcwilliams_new_2012}{2012}), Einav, Jenkins, and
Levin (\citeproc{ref-einav_contract_2012}{2012}), Einav, Jenkins, and
Levin (\citeproc{ref-einav_impact_2013}{2013}), Brown et al.
(\citeproc{ref-brown_how_2014}{2014}), Liberman et al.
(\citeproc{ref-liberman_equilibrium_2018}{2018}), Nelson
(\citeproc{ref-nelson_private_2025}{2025}), Einav, Finkelstein, and
Mahoney (\citeproc{ref-einav_io_2021}{2021})). By studying the
transition to risk-based premiums for environmental liability, this
research contributes new evidence on how mispriced insurance can distort
markets, particularly in hazardous industries where the social costs of
pollution are significant.

This paper also contributes to the literature on U.S. UST policy. While
much of the existing work focuses on estimating the marginal damages of
pollution from leaks (Marcus (\citeproc{ref-marcus_going_2021}{2021}),
Zabel and Guignet (\citeproc{ref-zabel_hedonic_2012}{2012}), Guignet
(\citeproc{ref-guignet_sell_2014}{2014}), Guignet and Martinez-Cruz
(\citeproc{ref-guignet_impacts_2018}{2018}), Guignet et al.
(\citeproc{ref-guignet_contamination_2018}{2018}), Walsh and Mui
(\citeproc{ref-walsh_contaminated_2017}{2017}), Guignet
(\citeproc{ref-guignet_what_2013}{2013})), this study is complementary,
examining the design of management policies intended to prevent those
damages.

This research is most closely related to studies of a similar insurance
market reform in Michigan by Yin, Pfaff, and Kunreuther
(\citeproc{ref-yin_can_2011}{2011}), Yin, Kunreuther, and White
(\citeproc{ref-yin_risk-based_2011}{2011}), and Yin, Kunreuther, and
White (\citeproc{ref-yin_environmental_2007}{2007}). That body of work
finds that transitioning to risk-based insurance reduced the number of
small firms and reported leaks. However, those analyses relied on
aggregate or cross-sectional data, which cannot control for unobserved
facility-level heterogeneity and may be susceptible to omitted variable
bias.This research is most closely related to studies of a similar
insurance market reform in Michigan by Yin, Pfaff, and Kunreuther
(\citeproc{ref-yin_can_2011}{2011}), Yin, Kunreuther, and White
(\citeproc{ref-yin_risk-based_2011}{2011}), and Yin, Kunreuther, and
White (\citeproc{ref-yin_environmental_2007}{2007}). That body of work
finds that transitioning to risk-based insurance reduced the number of
small firms and reported leaks. However, those analyses relied on
aggregate or cross-sectional data, which cannot control for unobserved
facility-level heterogeneity and may be susceptible to omitted variable
bias. This paper advances on prior work by adopting an empirical
strategy similar to other modern studies of environmental liability
(Boomhower (\citeproc{ref-boomhower_drilling_2019}{2019})). By using a
rich, facility-level panel dataset, I can implement
difference-in-differences models that control for unobserved facility
characteristics and common time shocks, providing more credible causal
estimates of how firms adjust their technology, replacement, and exit
decisions in response to cost-based pricing.

\section{Setting and Data}\label{setting-and-data}

\subsection{The Policy and Technical Landscape of U.S. Underground
Storage
Tanks}\label{the-policy-and-technical-landscape-of-u.s.-underground-storage-tanks}

An underground storage tank (UST) is defined by the U.S. Environmental
Protection Agency as any tank, including its connected underground
piping, with at least 10\% of its volume below ground (EPA
(\citeproc{ref-epa_underground_2024}{2024})). Most UST facilities house
multiple tanks on a single site and typically contain hazardous
substances like petroleum. Importantly, the majority of these facilities
are retail gas stations. In Texas, for example, an annual average of
75\% of active facilities during the sample period are retail gas
stations. The next largest category is fleet refueling firms (20\%),
with the remainder comprising airports, marinas, industrial plants, and
municipal properties. These shares are consistent with national
averages, where retail gas stations account for approximately 75\% of
all UST facilities (EPA (\citeproc{ref-epa_underground_2024}{2024})).

The primary technical distinction between USTs is their construction:
either single-walled or double-walled. This initial technology choice is
critical because, in practice, facility owners rarely invest in
incremental upgrades once a tank is operational. Comprehensive data on
such improvements are sparse, but evidence from Texas, where pricing
incentives to upgrade have been strongest, is telling. From 1998 to
2021, only 0.003\% of facilities ever upgraded their tanks. Instead,
investment decisions are concentrated at two points: initial
installation and subsequent closure with tank replacement, or complete
exit from the market. This resistance to midlife retrofits underscores
their high cost and reflects the perverse incentives created by
vintage-differentiated minimum performance standards, which often
grandfather in older, less safe technologies.\footnote{For example, 30
  Tex. Admin. Code §334.45(d)(E)(i) requires that new tanks and piping
  (installed on or after Jan.~1, 2009) ``must incorporate secondary
  containment''. In other words, only newly installed UST systems must
  have dual containment, whereas older tanks (installed before that
  date) are not retroactively required to do so. Furthermore, the Texas
  code defines a new UST system as one whose installation ``commenced
  after December 22, 1988.'' Every state has similar language,
  effectively grandfathering in older tanks and creating a
  vintage-differentiated regulation that discourages retrofits
  Gruenspecht (\citeproc{ref-gruenspecht_differentiated_1982}{1982})}

A facility's choice of UST construction carries significant economic and
environmental consequences. Single-walled tanks are markedly cheaper to
build but leak far more often than double-walled designs. Retail
catalogues and direct correspondence with a major manufacturer indicate
that single-walled fiberglass tanks sell for roughly one-half to
one-third the price of comparable double-walled units.\footnote{Direct
  correspondence with one of the largest UST manufacturers confirm that
  single-walled tank costs remains roughly one-half to two-thirds the
  price of an otherwise comparable double-walled tank. The sales team
  (speaking under a promise of anonymity) explained that they update
  list prices to reflect general input-cost inflation, i.e with general
  input costs, but the manufacturer deliberately keeps the percentage
  surcharge for secondary containment almost unchanged. Historical
  evidence echoes this pattern. An industry survey reported by the U.S.
  GAO priced a new 10 000-gallon fiberglass tank in 1987 at about \$23
  000 for single wall versus \$39 000 for double wall, a 67 percent
  premium that has persisted for decades ({``Hazardous Materials:
  Upgrading of Underground Storage Tanks Can Be Improved to Avoid Costly
  Cleanups''} (\citeproc{ref-gao1987}{1992})). A January 2021
  Containment Solutions guide lists double-walled fiberglass tanks from
  \$162 270 (15 000 gal) to \$462 425 (50 000 gal); although the
  catalogue no longer offers single-wall primaries, accessory pages show
  one-third lower prices for single-wall collars and sumps, implying a
  similar shell-level surcharge (Containment Solutions, Inc.
  (\citeproc{ref-csi2021}{2021})). A 2018 Xerxes list quotes single-wall
  prices of \$19 382 (10 000 gal) and \$174 690 (50 000 gal), while the
  matching double-wall prices are \$47 497 and \$326 671, confirming a
  60--90 percent differential in that year (Xerxes Corporation
  (\citeproc{ref-xerxes2018}{2018})). All figures cover the tank itself;
  installation labour, required release-detection equipment, and other
  site-specific costs are additional.} This cost-safety trade-off is
borne out empirically. As shown in Figure~\ref{fig-leak-risk},
facilities equipped with single-walled systems are significantly more
likely to report releases than their double-walled peers at every age.

The evolution of the nation's UST stock has been heavily shaped by the
1998 federal deadline requiring the closure or upgrade of all
non-compliant tanks. In the years leading up to this deadline, the
average annual closure rate was a substantial 25.56\% (SE = 9.91).
Following this large-scale removal of older tanks, the turnover rate
dropped dramatically; from 1999 onward, the average annual closure rate
fell to just 3.29\% (SE = 3.12). This slow post-deadline replacement
cycle has led to a persistently old and high-risk tank population. In
1994, single-wall tanks constituted 56.2\% of all active tanks, and
while this share has declined, they still represented 47.8\% of the
active stock in 2019. This persistence is reflected in the age of the
tanks, with the average age of an active tank steadily increasing from
13.3 years in 1994 to 30.7 years in 2019. Across the entire sample,
single-wall tanks are significantly older, with an average age of 22.5
years, compared to 11.7 years for double-wall tanks. While the 1998
deadline removed a large cohort of the riskiest tanks, the subsequent
low turnover rate has resulted in a significant and aging stock of
high-risk USTs remaining in operation today.

Given the significant cleanup costs from leaks, federal regulations
require UST owners to demonstrate financial responsibility, typically by
securing \$1 million in coverage. In the early 1990s, with the private
insurance market for this risk still nascent, many states established
UST Trust Funds to provide publicly managed insurance pools. However,
unlike traditional insurance, these public funds typically operated on a
flat-fee basis, charging uniform premiums regardless of
facility-specific risk. While this approach ensures remediation costs
are covered, it creates a significant price externality. By pooling
diverse risks under one price, these funds distort incentives for risk
mitigation and crowd out the private insurance market. In states with
such funds, they consistently cover 90-100\% of all facilities, an
unsurprising outcome given that the subsidized, flat-fee contracts are
difficult for any private, risk-based insurer to compete
against.\footnote{Surveys from the Association of State and Territorial
  Solid Waste Management Officials (ASTSWMO
  (\citeproc{ref-astswmo_archive}{2025})) show that between 2000 and
  2021 consistently show that in states with public funds, these funds
  cover 90-100\% of facilities. This holds true even in state fund
  states that permit owners to use private insurance, indicating the
  state fund is the dominant instrument.}

Six states have since transitioned away from this public model to
private insurance markets, with Texas transitioning in 1999. This shift
provides the setting for this paper's analysis. In Texas, from 2007 to
2021, an average of 89.8\% of retail UST facilities met their financial
responsibility requirements using private insurance. The remaining
10.2\% relied on other mechanisms, predominantly self-insurance, a path
available only to large owners who can meet stringent financial asset
tests. Crucially, this private insurance market appears to incorporate
facility risk. On average, 69.4\% of these privately insured facilities
were comprised of at least one single-wall tank. This indicates that
higher-risk facilities are not excluded from the market but instead
select into private insurance, where they face risk-adjusted premiums
that reflect their specific operational characteristics.

In summary, the U.S. is managing a large and aging stock of underground
storage tanks where construction type is a strong predictor of leak
risk. The prevailing insurance regime, either a flat-fee public fund or
a cost-based pricing private market, critically shapes the incentives
for managing these risks. To empirically test how insurance design
affects firm behavior, the following sections detail the data and causal
methodology used to analyze Texas's transition to a private insurance
market.

\clearpage

\section{Setting, Data, and Institutional
Background}\label{setting-data-and-institutional-background}

\subsection{The Technical and Regulatory Landscape of U.S. Underground
Storage
Tanks}\label{the-technical-and-regulatory-landscape-of-u.s.-underground-storage-tanks}

The physical setting of this analysis concerns the U.S. Environmental
Protection Agency (EPA) definition of an underground storage tank (UST)
system: any tank and connected underground piping with at least 10\% of
its volume beneath the surface (EPA
(\citeproc{ref-epa_underground_2024}{2024})). These facilities are
predominantly retail gas stations, which account for approximately 75\%
of active sites in both the national average and the Texas-specific
sample. The remainder comprises fleet refueling depots (20\%) and a mix
of industrial, aviation, and municipal facilities.

The primary source of heterogeneity in this capital stock is the tank
construction technology. The critical distinction lies between
single-walled systems---essentially bare steel or fiberglass
shells---and double-walled systems, which incorporate an interstitial
space for leak detection. This initial technology choice is economically
critical because facility owners rarely invest in incremental upgrades
once a tank is operational. From 1998 to 2021, only 0.003\% of Texas
facilities ever retrofitted their tanks prior to a terminal closure
event. Investment decisions are consequently concentrated at the
extensive margin of entry and exit. This resistance to midlife retrofits
reflects high switching costs and the perverse incentives created by
vintage-differentiated standards. State regulations typically
grandfather in older systems, requiring secondary containment only for
new installations initiated after specific cutoff dates, such as
December 1988 or January 2009.

The construction type dictates the environmental risk profile.
Single-walled tanks are significantly cheaper to install but exhibit
higher failure rates. Historical industry data indicates a persistent
price differential; a 1987 GAO survey priced single-walled units at
approximately \$23,000 compared to \$39,000 for double-walled
equivalents, a 67\% premium. Federal regulation explicitly targets this
risk through the 1984 Hazardous and Solid Waste Amendments (HSWA), which
culminated in a 1998 deadline requiring the closure or upgrade of
non-compliant tanks. While the years preceding this deadline saw annual
closure rates averaging 25.56\% (SE = 9.91), the post-deadline turnover
dropped precipitously to 3.29\% (SE = 3.12). This collapse in capital
turnover resulted in an aging legacy fleet. In 2019, single-walled tanks
still constituted 47.8\% of the active stock, with an average age of
22.5 years compared to 11.7 years for double-walled units.

\subsection{Financial Responsibility and the Insurance
Regime}\label{financial-responsibility-and-the-insurance-regime}

To mitigate the liability associated with these environmental risks, the
EPA's 1988 Financial Responsibility Rule (40 CFR Part 280, Subpart H)
mandates that owners secure coverage for corrective action and
third-party liability, typically \$1 million per occurrence. In the late
1980s, the immaturity of the private environmental insurance market led
many states to establish public trust funds. These funds typically
operate on a flat-fee basis, charging uniform premiums regardless of
facility-specific risk. While effective at ensuring solvency, this
pooling mechanism creates a price externality that distorts incentives
for risk mitigation. In states retaining such funds, they consistently
cover 90\% to 100\% of facilities.

Texas initially utilized a public fund model but diverged in 1998
following the passage of House Bill 2587. Facing actuarial insolvency,
the legislature sunset the Texas Petroleum Storage Tank Remediation
Fund, forcing the entire facility population to migrate to the private
market or self-insure by 1999. This sharp policy discontinuity serves as
the central identification strategy. In the post-transition period
(2007--2021), 89.8\% of Texas retail facilities met their obligations
through private insurance, with the remainder relying on self-insurance
mechanisms available only to highly capitalized firms.

\section{Conceptual Framework: Insurance Design and Optimal
Stopping}\label{conceptual-framework-insurance-design-and-optimal-stopping}

The facility owner's decision is modeled as a single-agent dynamic
discrete choice problem following the optimal stopping literature (Rust,
1987). In an infinite horizon setting with time indexed by \(t\), the
agent operates a single underground storage tank (UST) and observes the
state variable \(a_t\), representing the tank's age. At the beginning of
each period, the agent chooses a binary action
\(d_t \in \{ \text{Maintain}, \text{Close} \}\) to maximize the expected
present discounted value of future payoffs. If the agent chooses to
close (\(d_t = \text{Close}\)), the facility exits the market
permanently, receiving a terminal scrap value \(\kappa\) representing
the liquidity of the land net of decommissioning costs. Conditional on
maintenance (\(d_t = \text{Maintain}\)), the flow utility is specified
as \(u(a_t) = R - P_j(a_t)\), where \(R\) denotes the annual net revenue
from operations less all non-insurance costs, and \(P_j(a_t)\) is the
insurance premium under regime \(j\). While the structural estimation in
Section 5 relaxes this assumption to incorporate deductibles and partial
risk internalization, this simplified framework assumes full insurance
coverage for tractability, treating environmental risk solely as an
ex-ante premium cost rather than an ex-post liability shock. The agent's
value function \(V(a)\) satisfies the Bellman equation:

\[
V(a) = \max \left\{ R - P_j(a) + \beta \mathbb{E}[V(a') \mid a], \kappa \right\}
\]

where \(\beta \in (0, 1)\) is the discount factor and the expectation is
taken over the stochastic evolution of tank age. The optimal stopping
rule dictates that the firm closes the tank when the continuation value
of operation falls below the scrap value \(\kappa\).

The divergence in firm behavior across states is driven by the
elasticity of the premium function \(P_j(a)\) with respect to capital
depreciation. Under the \textbf{Uniform Premium (Public Fund)} regime,
the price is invariant to risk, such that \(P(a) = \bar{P}\) and the
marginal cost of aging is zero (\(\frac{\partial P}{\partial a} = 0\)).
This flat pricing structure eliminates the financial incentive to retire
aging capital, pivoting the continuation value curve upward and causing
the firm to retain the asset until it is physically obsolete, defined as
age \(a^*_{UP}\). Conversely, under the \textbf{Risk-Based (Private
Market)} regime, premiums are strictly increasing in the leak hazard
rate \(h(a)\), such that \(\frac{\partial P}{\partial a} > 0\). This
introduces a positive marginal cost of aging, pivoting the value
function downward for older tanks and accelerating the optimal exit age
to \(a^*_{RB}\). Figure 1 illustrates this mechanism, demonstrating that
the introduction of risk-based pricing strictly reduces the optimal tank
lifespan relative to the uniform pricing baseline
(\(a^*_{RB} < a^*_{UP}\)).

\begin{figure}

\caption{\label{fig-mechanism}The Mechanism: Optimal Stopping under
Alternative Insurance Regimes}

\centering{

\includegraphics[width=0.9\textwidth,height=\textheight]{Output/Figures/Paper/Fig_1_Mechanism.pdf}

}

\end{figure}%

While risk-based pricing improves allocative efficiency by penalizing
higher failure rates, it does not achieve the first-best social optimum
due to unpriced health externalities. The social planner's objective
function includes both the remediation costs (\(L\)) covered by insurers
and the health damages (\(E\)) from groundwater contamination, which are
typically unobservable and generate no third-party claims (Marcus,
2021). Consequently, the private actuarial premium
\(P_{RB} \propto h(a)L\) is strictly less than the social marginal cost
\(h(a)(L+E)\). This wedge implies that while the private market induces
earlier exit than the public fund, the equilibrium exit age \(a^*_{RB}\)
remains higher than the socially optimal age \(a^*_{SOC}\). Figure 2
visualizes this welfare loss, identifying the shaded region between
\(a^*_{SOC}\) and \(a^*_{RB}\) as the uninternalized externality. The
theoretical prediction for the empirical analysis is therefore a
monotonic ordering of tank lifespans where
\(a^*_{SOC} < a^*_{RB} < a^*_{UP}\).

\begin{figure}

\caption{\label{fig-welfare}Welfare Analysis: The Externality Wedge and
Partial Correction}

\centering{

\includegraphics[width=0.9\textwidth,height=\textheight]{Output/Figures/Paper/Fig_2_Welfare.pdf}

}

\end{figure}%

\subsection{Data and Sample Construction}\label{sec-data}

The empirical analysis relies on a composite panel constructed from
three distinct layers of administrative microdata. These layers comprise
a comprehensive facility-level backbone covering the universe of
regulated entities, a market-based premium overlay for the treated
jurisdiction of Texas, and a claims-cost overlay for a subset of control
states.

\subsubsection{The Administrative
Backbone}\label{the-administrative-backbone}

The core dataset consists of the universe of regulated UST facilities in
Texas and eighteen control states, harmonized from raw state-level
administrative inventories and EPA summary files. The panel tracks
facility lifecycles from initial registration through 2023. For every
facility-month observation, I observe the exact timing of critical
behavioral outcomes including tank closures, system retrofits, and
confirmed releases. As shown in the gray intervals of
Figure~\ref{fig-d1} (Panel B), this administrative backbone is
continuous across the 1999 policy intervention window for all sampled
states. This continuity ensures that the primary outcome variables
(\(Y_{it}\)) are measured consistently before and after the treatment.

\begin{figure}

\caption{\label{fig-d1}Analysis Sample Coverage. Administrative facility
data exists for all analysis states; Cost data is limited to specific
subsets.}

\begin{minipage}{\linewidth}
\begin{center}
\includegraphics[width=0.95\textwidth,height=\textheight]{Output/Figures/FD01_analysis_coverage_combined.png}
\end{center}
\end{minipage}%
\newline
\begin{minipage}{\linewidth}
\small \emph{Note:} Notes: Gray bars in Panel B represent the `Admin
Data' backbone (facility entry/exit/leaks) available for all analysis
states. Colored overlays indicate periods where cost data is available:
Orange for Texas Market Premiums (starting 2006), and Blue for cleanup
costs in select control states. The gaps illustrate the data matching
challenge.\end{minipage}%

\end{figure}%

\subsubsection{Premium Reconstruction and Rating Engine
Dynamics}\label{premium-reconstruction-and-rating-engine-dynamics}

To parameterize the structural model, I reconstruct the private
insurance premiums that Texas facilities faced following the 1999
privatization. Because private insurers do not publicly disclose
individual policy premiums, I build a ``bottom-up'' rating engine based
on the actuarial filings of the Mid-Continent Casualty Company, a
dominant insurer in the Texas UST market. Using rate manuals obtained
from the System for Electronic Rates \& Forms Filing (SERFF) covering
2006 through 2021, I recover the precise algorithms used to price tank
risk. I estimate the monthly premium \(P_{ijt}\) for tank \(j\) at
facility \(i\) in month \(t\) using the insurer's canonical pricing
formula:

\[
P_{ijt} = \text{Base}_{t} \times \text{ILF}_{it} \times \left( 1 + \sum_{k} \lambda_{k,t} \mathbf{X}_{ijt} \right)
\]

The resulting pricing engine exhibits four distinct structural
properties verified in the rate filings. First, the insurer prices
depreciation through a discrete step function rather than a linear
trend. As shown in Figure~\ref{fig-age-steps}, premiums remain
relatively flat for new capital but ratchet up significantly as tanks
cross specific actuarial age thresholds (e.g., 20 and 30 years),
creating non-linear incentives for replacement.

\begin{figure}

\caption{\label{fig-age-steps}The Age Pricing Schedule (Step Function).
Visualizing the discrete age-loading tables from the rate filings
relative to tank vintage.}

\begin{minipage}{\linewidth}
\begin{center}
\includegraphics[width=0.8\textwidth,height=\textheight]{Output/Figures/FD02a_engine_age_steps.png}
\end{center}
\end{minipage}%
\newline
\begin{minipage}{\linewidth}
\small \emph{Note:} The solid line represents the mean per-tank premium
calculated by the actuarial engine for each integer age (0--40 years).
The discrete steps correspond to the age bins defined in the
Mid-Continent rate filings (e.g., 0--5, 6--10, 11--15). The gray bars
(right axis) show the distribution of tank ages in the Texas sample,
indicating that a significant portion of the fleet is in the older,
higher-priced age brackets.\end{minipage}%

\end{figure}%

Second, conditional on age, the construction technology dictates the
base risk load. Figure~\ref{fig-risk-loads} illustrates the additive
premium surcharges applied to single-walled and steel systems.
Single-walled tanks face substantial penalties independent of their age,
reflecting the insurer's assessment of their inherent structural
vulnerability.

\begin{figure}

\caption{\label{fig-risk-loads}Construction Risk Loads. Impact of
`construction\_load' and `piping\_load' factors.}

\begin{minipage}{\linewidth}
\begin{center}
\includegraphics[width=0.8\textwidth,height=\textheight]{Output/Figures/FD02b_engine_risk_loads.png}
\end{center}
\end{minipage}%
\newline
\begin{minipage}{\linewidth}
\small \emph{Note:} Boxplots showing the distribution of per-tank
premiums by tank technology type. ``Double-Walled'' tanks receive the
lowest base rates (baseline risk). ``Fiberglass'' and
``Single-Walled/Steel'' tanks incur additive percentage surcharges (risk
loads) as specified in the rate filings, resulting in strictly higher
premiums. Diamonds indicate group means.\end{minipage}%

\end{figure}%

Third, the market structure suggests constant returns to scale regarding
liability. As demonstrated in Figure~\ref{fig-scale-check}, the rating
engine applies no volume discounts; the per-tank premium remains
invariant to the number of tanks at a facility. This implies that larger
facilities face a linear cost function with respect to their capital
stock size.

\begin{figure}

\caption{\label{fig-scale-check}Scale Neutrality Check. Engine Logic: No
explicit volume discount (Flat per-unit price expected).}

\begin{minipage}{\linewidth}
\begin{center}
\includegraphics[width=0.8\textwidth,height=\textheight]{Output/Figures/FD02c_engine_scale_check.png}
\end{center}
\end{minipage}%
\newline
\begin{minipage}{\linewidth}
\small \emph{Note:} Average per-tank premium plotted against the number
of tanks at the facility. The flat trend confirms that the pricing
engine aggregates risk linearly (sum of individual tank premiums)
without applying bulk discounts for larger facilities. This validates
the use of per-tank pricing as the fundamental unit of
analysis.\end{minipage}%

\end{figure}%

Finally, the insurer actively updates these schedules to reflect
evolving risk perceptions. Figure~\ref{fig-filing-evolution} decomposes
the aggregate age curves by regulatory filing era. Over time, the
insurer has steepened the penalty for older tanks, particularly in the
2019 and 2021 filings, indicating a learning process where the realized
losses from the aging legacy fleet are increasingly priced into the
market.

\begin{figure}

\caption{\label{fig-filing-evolution}Evolution of the Age-Risk Pricing
Schedule. Each line represents the active actuarial age curve during a
specific filing period.}

\begin{minipage}{\linewidth}
\begin{center}
\includegraphics[width=0.95\textwidth,height=\textheight]{Output/Figures/FE_age_curve_by_filing.png}
\end{center}
\end{minipage}%
\newline
\begin{minipage}{\linewidth}
\small \emph{Note:} This figure decomposes the aggregate age curve by
regulatory filing era. It visualizes how the insurer's assessment of
age-related risk has evolved, showing the specific step-function
schedules active during the 2006--2014, 2014--2019, 2019--2020, and
post-2021 periods.\end{minipage}%

\end{figure}%

\subsubsection{Addressing Data Gaps}\label{addressing-data-gaps}

The analysis addresses two specific dimensions of data sparsity
visualized in Figure~\ref{fig-d1}. First, regarding geographic
heterogeneity, the cost estimation routine is restricted to control
states with verifiable claims data and extrapolated to
administrative-only states based on observable facility characteristics.
Second, regarding temporal gaps, the administrative backbone extends
beyond the cost data windows. Specifically, the period from 1999 to 2005
in Texas represents a transition window where private prices existed but
granular microdata is unavailable. For such periods, I assume
stationarity in the conditional cost distribution while allowing for
aggregate inflation adjustments.

\subsubsection{Cost Data Overlay: Control Group
Liability}\label{cost-data-overlay-control-group-liability}

While the treated group faces ex-ante premium pricing, the control group
operates under a public insurance regime where costs are realized
ex-post. To parameterize the severity of environmental failure
(\(C \mid \text{Leak}\)) for these facilities, I link administrative
facility identifiers to the payment ledgers of state trust funds in
Colorado, Louisiana, New Mexico, Tennessee, and Utah. This subset
provides the empirical distribution of cleanup costs required to model
the ``state's burden'' under the flat-fee regime.

The resulting cost distribution, visualized in
Figure~\ref{fig-claims-dist}, is characterized by extreme positive
skewness. The data rejects a normal distribution hypothesis in favor of
a heavy-tailed log-normal structure. The distinction between the median
and mean cost is economically significant; while the median represents
the typical remediation event, the mean is driven upward by catastrophic
outliers---rare events where groundwater contamination plumes migrate
off-site, triggering liability magnitudes orders of magnitude larger
than the modal claim.

\begin{figure}

\caption{\label{fig-claims-dist}Distribution of Realized Cleanup Costs.
The distribution is right-skewed with a heavy tail of catastrophic
damages.}

\begin{minipage}{\linewidth}
\begin{center}
\includegraphics[width=0.95\textwidth,height=\textheight]{Output/Figures/FD03a_claims_dist.png}
\end{center}
\end{minipage}%
\newline
\begin{minipage}{\linewidth}
\small \emph{Note:} Notes: This histogram plots the frequency of Leaking
Underground Storage Tank (LUST) cleanup costs across the five control
states (CO, LA, NM, TN, UT), adjusted to 2023 dollars. The x-axis uses a
log scale. The dashed black line marks the median, while the dotted red
line marks the mean. The substantial deviation between mean and median
illustrates the heavy-tailed nature of environmental
liability.\end{minipage}%

\end{figure}%

Furthermore, liability exhibits spatial heterogeneity driven by
differences in hydrogeology and state-specific remediation standards.
Figure~\ref{fig-claims-state} breaks down these costs by jurisdiction.
Although the central tendency (median) varies by state, the
dispersion---specifically the reach of the upper tail---remains a
consistent feature across all sampled public funds. This confirms that
the ``catastrophic tail risk'' is a fundamental technological property
of UST systems rather than an artifact of specific state administrative
procedures.

\begin{figure}

\caption{\label{fig-claims-state}Realized Cleanup Costs by State.
Heterogeneity in environmental liability magnitude across control
jurisdictions.}

\begin{minipage}{\linewidth}
\begin{center}
\includegraphics[width=0.95\textwidth,height=\textheight]{Output/Figures/FD03b_claims_by_state.png}
\end{center}
\end{minipage}%
\newline
\begin{minipage}{\linewidth}
\small \emph{Note:} Notes: Boxplots display the distribution of cleanup
costs (log scale, 2023 dollars) for each state in the claims sample. The
central line represents the median cost, while the box captures the
interquartile range. While median costs vary by jurisdiction, the
presence of high-cost outliers is a consistent feature across all state
funds.\end{minipage}%

\end{figure}%

\subsubsection{Cost Data Overlay and Gap
Adjustments}\label{cost-data-overlay-and-gap-adjustments}

To parameterize the structural model, I overlay financial data onto the
administrative backbone as depicted by the colored intervals in
Figure~\ref{fig-d1}. Post-1999, Texas facilities migrated to private
insurance where I observe facility-specific premium quotes and coverage
terms from the \emph{Mid-Continent Group} rate filings. While the policy
shock occurred in 1999, granular premium microdata is only available
starting in 2006. The period from 1999 to 2005 represents a transition
window where prices existed but are not directly observed in the sample.
For the control group, I link facility identifiers to state fund claim
records to derive the empirical distribution of cleanup costs
conditional on a leak.

The analysis addresses two specific dimensions of data sparsity
visualized in Figure~\ref{fig-d1}. First, regarding geographic
heterogeneity, not all control states provide claims transparency. The
analysis therefore restricts the cost estimation routine to states with
verifiable claims data and extrapolates these cost functions to
administrative-only states based on observable facility characteristics
such as age, tank capacity, and local geology. Second, regarding
temporal gaps, the administrative backbone extends beyond the cost data
windows. For periods where cost data is missing, such as the Texas 1999
to 2005 window or pre-1990 control states, I assume stationarity in the
conditional cost distribution while allowing for aggregate inflation
adjustments. This structure implies that while firm behavior (\(Q\)) is
observed continuously, financial incentives (\(P\)) are imputed during
gap years using the structural parameters recovered from the data-rich
windows.

\subsubsection{Facility Risk and Endogenous
Detection}\label{facility-risk-and-endogenous-detection}

The final data layer characterizes the realized environmental
performance of the tank capital, specifically how the insurance regime
alters the \emph{timing} of leak discovery. I define the annual leak
probability using the realized history of Leaking Underground Storage
Tank (LUST) incidents reported to state regulators. Aggregating the
monthly panel to the facility-year level, I calculate the realized
failure rate per effective exposure-year.

Crucially, I distinguish between two mutually exclusive detection
mechanisms. \textbf{Operational Discovery} refers to releases identified
by leak detection equipment (e.g., automatic tank gauging, interstitial
monitoring) while the facility is active. \textbf{Closure Discovery}
refers to releases identified only during the physical removal of the
tank system.

Figure~\ref{fig-leak-prob} reveals a structural shift in detection
behavior driven by the insurance regime. In the Control Group (State
Funds), a significant proportion of releases are ``Found at Closure''
(Orange), indicating a passive monitoring environment where leaks
accumulate as latent liabilities, hidden until market exit. In contrast,
the Texas (Private Insurance) sample exhibits a higher rate of
``Operational Discovery'' (Blue) and a collapsed rate of closure-found
leaks. This shift is consistent with the presence of active insurer
monitoring: private underwriters require proof of integrity for
coverage, forcing the endogenous detection of leaks that would otherwise
remain hidden under the flat-fee regime.

\begin{figure}

\caption{\label{fig-leak-prob}Annual Leak Probability by Detection
Method. Decomposing environmental risk into operational failures and
latent liabilities revealed at exit.}

\begin{minipage}{\linewidth}
\begin{center}
\includegraphics[width=0.8\textwidth,height=\textheight]{Output/Figures/FD04a_leak_annual_prob.png}
\end{center}
\end{minipage}%
\newline
\begin{minipage}{\linewidth}
\small \emph{Note:} Notes: This stacked bar chart displays the
unconditional annual probability of a confirmed release for facilities
in the analysis sample. The ``Operational Discovery'' (Blue) component
represents leaks detected while the tank was in service. The ``Found at
Closure'' (Orange) component represents leaks detected during tank
removal or permanent closure. The height of the bar represents the total
annual hazard rate. The shift from ``Closure'' to ``Operational''
detection in Texas illustrates the monitoring effect of private
insurance.\end{minipage}%

\end{figure}%

\section{Causal Evidence from the Transition to Private
Insurance}\label{causal-evidence-from-the-transition-to-private-insurance}

This section presents the empirical evidence on the effects of
risk-adjusted insurance on UST facilities by analyzing the behavioral
responses of incumbent facilities to the transition from flat-fee public
insurance to risk-based private market coverage. The analysis focuses on
identifying causal effects through comparison of treatment and control
states, examining how facilities adjusted their technology adoption,
closure decisions, and risk management practices when faced with
insurance premiums that reflected their actual environmental risk
profile.

\subsection{The Causal Impact of Risk-Based Insurance on Firm
Behavior}\label{the-causal-impact-of-risk-based-insurance-on-firm-behavior}

This section presents the main empirical results of the paper, using a
difference-in-differences framework to compare Texas with control states
that maintained public insurance funds. The analysis proceeds in two
steps. First, to establish the economic rationale for risk-based
pricing, I demonstrate that observable facility characteristics, such as
tank age and construction type, are strong and significant predictors of
environmental risk. This confirms that a key condition for an efficient
insurance market is met. Second, having established this predictive
relationship, I then estimate the causal impact of the transition to
risk-based insurance on the behavior of incumbent UST facilities,
focusing on their decisions to close, replace, or exit the market.
Finally, I examine whether these behavioral changes had a measurable
impact on environmental outcomes, specifically reported leak rates. This
two-step approach allows for a robust analysis of the effects of
risk-based insurance on both firm behavior and environmental risk
mitigation.

\subsubsection{Environmental Risk and UST Facility
Characteristics}\label{environmental-risk-and-ust-facility-characteristics}

Risk-rated insurance markets can significantly reduce environmental
hazards by aligning private facility incentives with social costs---a
critical mechanism for internalizing pollution externalities. When
insurance premiums accurately reflect facility-specific risk factors,
owners face direct financial incentives to invest in safer technologies
and operational practices. However, the effectiveness of this
market-based approach hinges entirely on insurers' ability to accurately
assess facility-level risk characteristics. Without precise risk
assessment, the transition from flat-fee to cost-based pricing contracts
may fail to improve environmental outcomes or could even create perverse
incentives for underreporting and maintenance avoidance. To quantify the
predictive relationship between observable facility characteristics and
actual environmental risk, I employ a 10-fold cross-validation approach
using a comprehensive facility-level panel dataset of USTs:

\begin{equation}\phantomsection\label{eq-risk-pred}{
\begin{aligned}
10\text{-fold cross-validation:} &\\
\text{Reported Leak}_{i,s,t} =& \alpha_t + \delta_s + \sum_{k} \beta_k \text{Wall}_{k,i} + \sum_{m} \gamma_m \text{Age Bins}_{m,i} + \\
&\sum_{k,m} \zeta_{km} (\text{Wall}_{k,i} \times \text{Age}_{m,i}) + \theta\mathbf{X}_{i,s,t} + \epsilon_{i,s,t}
\end{aligned}
}\end{equation}

In this cross-validation framework, I model the probability of a
reported leak as a function of facility characteristics while
controlling for temporal and spatial heterogeneity. The dependent
variable, \(\text{Reported Leak}_{i,s,t}\), is a binary indicator equal
to one if facility \(i\) in state \(s\) reports a leak in year \(t\).
The model includes year fixed effects (\(\alpha_t\)) and state fixed
effects (\(\delta_s\)) to account for temporal trends and state-specific
regulatory environments, respectively. To capture the relationship
between tank technology and leak risk, I include indicator variables for
single-walled UST systems (\(\text{Wall}_{k,i}\)) and categorize
facility age into 5-year bins (\(\text{AgeBin}_{m,i}\)). The interaction
terms between wall type and age bins
(\(\text{Wall}_{k,i} \times \text{Age}_{m,i}\)) allow for heterogeneous
effects of tank construction across different facility age profiles.
Additionally, the vector \(\mathbf{X}_{i,s,t}\) contains other
facility-specific characteristics that may influence leak probability,
including tank size, construction material, and the employed release
detection method. This specification enables the assessment of how
observable facility characteristics predict environmental risk while
controlling for unobserved heterogeneity across time and space.

Figure~\ref{fig-leak-risk} plots the predicted annual probability of a
reported leak by tank age and construction type, confirming that these
observable characteristics are strong predictors of environmental risk
and thus provide a clear economic rationale for risk-based pricing. The
two technologies exhibit starkly different hazard profiles. Single-wall
tanks (the orange line) follow an inverted U-shape, with the probability
of a reported leak rising sharply to a peak of nearly 1.8\% for tanks
aged 20-25 years before declining as the riskiest units are presumably
removed from service. In contrast, double-wall tanks (the green line)
have a much lower and flatter risk profile for most of their operational
life. The key takeaway is that single-wall tanks are substantially more
likely to have a reported leak than their double-wall counterparts
throughout the critical 5-to-30-year age window. This fundamental,
predictable difference in risk based on observable technology and age is
undeniable.

\begin{figure}

\caption{\label{fig-leak-risk}Predictive Relationship Between Facility
Characteristics and Reported Leak Probability.}

\begin{minipage}{\linewidth}
\begin{center}
\includegraphics[width=0.95\textwidth,height=\textheight]{C:/Users/kaleb/Box/UST-Insurance/Output/Figures/cv_leak_risk_abs_no_title.png}
\end{center}
\end{minipage}%
\newline
\begin{minipage}{\linewidth}
\small \emph{Note:} This figure shows predicted annual leak
probabilities by tank age and wall type, estimated using 10-fold
cross-validation with facility-level sampling. The model specification
includes wall type interacted with age bin categories, controlling for
active tanks, fuel type (gasoline/diesel), year, and state fixed
effects. Predictions are generated exclusively from out-of-sample
facilities in each iteration to reduce overfitting. Model performance
metrics: Overall RMSE = 0.0963, Single-Walled RMSE = 0.1063,
Double-Walled RMSE = 0.081. The total sample includes 4,191,898
facility-year observations with an overall leak rate of 1.13\%. Error
bars represent 95\% confidence intervals for the mean prediction across
all cross-validation iterations. Single-walled tanks consistently show
higher leak risk compared to double-walled tanks, with the risk gap
widening as tanks age, particularly after 15 years of
operation.\end{minipage}%

\end{figure}%

\subsection{Causal Evidence of Risk-Adjusted Insurance on UST Facility
Behavior}\label{causal-evidence-of-risk-adjusted-insurance-on-ust-facility-behavior}

This section presents the empirical evidence of the effect of the
transition to private market insurance on UST facility behavior. To
ensure clean identification of the causal impact, I restrict the
analysis to single-walled tanks installed before 1999. This sample
restriction serves multiple analytical purposes. First, single-walled
tanks represent the highest observable risk category, making them
particularly sensitive to insurance design changes and thus providing
the strongest signal for detection. Second, focusing exclusively on
pre-1999 installations eliminates potential selection effects from
facilities that entered after the policy announcement, which might
confound the treatment effect. Finally, this approach allows me to
isolate how incumbent facilities---those directly experiencing the
regulatory regime shift---modified their behavior in response to the
transition from flat-fee to risk-adjusted insurance premiums. By
examining this well-defined cohort, I can more precisely estimate the
treatment effect while minimizing bias from compositional changes in the
market.

To study the UST owner response I estimate a panel differnece in
differences model and the following event-study specifications to
identify the causal effect of the transition to private market insurance
on UST facility behavior:

\begin{equation}\phantomsection\label{eq-tank-closure-did}{
\text{Tank Closure}_{i,t} = \beta \cdot \text{Treatment}_{i,t} + \sum_{a=1}^{N} \gamma_a \cdot \mathbf{1}[\text{AgeBin}_a]_{i,t} + \alpha_i + \alpha_t + \epsilon_{i,t}
}\end{equation}

\begin{equation}\phantomsection\label{eq-tank-closure-event}{
\text{Tank Closure}_{i,t} = \sum_{k=-K, k \neq -1}^{L} \beta_k \cdot \mathbf{1}[t - t^*_i = k] \cdot \text{Texas}_i + \sum_{a=1}^{N} \gamma_a \cdot \mathbf{1}[\text{AgeBin}_a]_{i,t} + \alpha_i + \alpha_t + \epsilon_{i,t}
}\end{equation}

I estimate two empirical specifications to identify the causal effect of
transitioning to private market insurance on incumbent Texas UST Owners.
Equation \ref{eq-tank-closure-did} presents a standard
difference-in-differences specification where \(\text{Treatment}_{i,t}\)
is an indicator equal to one for Texas facilities in the post-1999
period. The coefficient \(\beta\) captures the average treatment effect
of the policy change. The specification includes facility fixed effects
(\(\alpha_i\)) that account for time-invariant facility characteristics
and year fixed effects (\(\alpha_t\)) that control for common temporal
shocks. I also include a flexible set of age indicators
(\(\mathbf{1}[\text{AgeBin}_a]_{i,t}\)) to account for differential
closure rates across the facility lifecycle.

Equation \ref{eq-tank-closure-event} presents an event study
specification that allows for time-varying treatment effects. The
coefficients \(\beta_k\) trace out the dynamic response to the policy
change, where \(k\) indexes time relative to the treatment year
\(t^*_i\). The omitted category is \(k=-1\) (the year immediately before
treatment), making all coefficients relative to this pre-treatment
period. This specification maintains the same controls as the
difference-in-differences model, allowing me to examine pre-trends for
parallel trends assumption validation and to investigate how the
treatment effect evolves over time.

\begin{table}

\caption{\label{tbl-reg_closure}\textbf{Policy effects on single-walled
facility decisions}.}

\centering{

\centering
\label{tbl:reg_closure}
\resizebox{\textwidth}{!}{
\begin{tabular}{lcccccc}
\toprule
\textbf{Dependent Variables:} & \multicolumn{2}{c}{\textbf{Tank Closure}} & \multicolumn{2}{c}{\textbf{Exit | Closure}} & \multicolumn{2}{c}{\textbf{Replace | Closure}} \\
\cmidrule(lr){2-3} \cmidrule(lr){4-5} \cmidrule(lr){6-7}
\textbf{Model:} & (1) & (2) & (3) & (4) & (5) & (6) \\
\midrule
Texas × Post-Policy & 0.0303$^{***}$ & 0.0183$^{***}$ & -0.0953$^{**}$ & -0.1107$^{***}$ & 0.0733 & 0.1032$^{**}$ \\
      & (0.0058) & (0.0054) & (0.0348) & (0.0345) & (0.0420) & (0.0361) \\
      \midrule
      Age Bin Controls        & No & Yes & No & Yes & No & Yes \\
      \midrule
      \textbf{Fixed-effects} & & & & & & \\
      Year FE             & Yes & Yes & Yes & Yes & Yes & Yes \\
      Facility FE         & Yes & Yes & Yes & Yes & Yes & Yes \\
      \bottomrule
      \multicolumn{7}{p{0.95\textwidth}}{\small \textit{Notes:}
      This table reports difference-in-differences estimates of Texas' 1999 transition from public to private UST insurance on single-walled tank facilities operating before 1999. All models include facility and year fixed effects. Columns 1-2 show that Texas facilities had higher tank closure rates (1.8-3.0 percentage points) after the policy change. Among facilities that closed tanks, Texas facilities were less likely to exit completely (columns 3-4) and more likely to replace tanks (columns 5-6). These results suggest risk-based insurance encouraged technology updates rather than facility abandonment. Standard errors (in parentheses) are clustered at the state level. Significance: $^{*}p<0.1$; $^{**}p<0.05$; $^{***}p<0.01$. Observations: 3,816,313 (columns 1-2), 136,078 (columns 3-6).

      }
      \end{tabular}
      }
      

}

\end{table}%

The results in Table~\ref{tbl-reg_closure} provide strong evidence that
the transition to risk-based insurance induced a significant behavioral
response from the owners of high-risk facilities. The positive and
highly significant coefficients in columns (1) and (2) show that the
policy increased the annual probability of closing a single-walled tank
in Texas by 1.8 to 3.0 percentage points relative to control states.
However, the subsequent columns reveal a crucial nuance in this
response. The negative and significant coefficients in columns (3) and
(4) indicate that, conditional on a closure, Texas facilities were 9.5
to 11.1 percentage points less likely to exit the market entirely.
Correspondingly, column (6) shows they were 10.3 percentage points more
likely to replace the tank. Taken together, these findings suggest that
risk-adjusted insurance did not simply drive high-risk firms out of the
market. Instead, it created a powerful incentive for technology
adoption, encouraging incumbent owners to scrap their riskiest assets
(single-walled tanks) and invest in safer replacements, thereby reducing
environmental risk while continuing operations.

\begin{figure}

\caption{\label{fig-close-event}Dynamic Policy Effects on Tank Closures
(Single-Wall Tanks, Texas)}

\begin{minipage}{\linewidth}
\begin{center}
\includegraphics[width=1\textwidth,height=\textheight]{C:/Users/kaleb/Box/UST-Insurance/Output/Figures/event_study_tank_closures.png}
\end{center}
\end{minipage}%
\newline
\begin{minipage}{\linewidth}
\small \emph{Note:} TThis figure presents an event study of the
differential effect of Texas' 1999 transition from public to private
insurance on underground storage tank (UST) closure decisions. The plot
shows point estimates and 95\% confidence intervals for the interaction
between a Texas indicator and each event year, relative to 1998 (the
year before the policy change). The sample is restricted to facilities
that had only single-walled tanks in the pre-1999 period. The underlying
regression includes facility and year fixed effects, with standard
errors clustered at the state level. The specification controls for a
1998 spike in tank removals that occurred industry-wide. The pattern
reveals how the policy affected the timing and magnitude of tank closure
decisions, with particular attention to whether facility owners
strategically accelerated or delayed tank closures in response to the
policy change. Estimates to the right of the vertical red line represent
post-policy treatment effects, while estimates to the left establish the
pre-trends necessary for causal identification in the
difference-in-differences framework.\end{minipage}%

\end{figure}%

The event study in Figure~\ref{fig-close-event} complements the
regression results by visualizing the dynamic impact of the policy on
single-walled tank closures and testing the crucial parallel trends
assumption. The coefficients for the pre-treatment period (years prior
to 1999) are all statistically insignificant and centered around zero.
This provides strong visual support for the parallel trends assumption,
indicating no systematic differences in closure trends between Texas and
control states before the policy change. Following the transition to
private insurance in 1999 (year 0), there is an immediate and sustained
increase in the probability of tank closure in Texas relative to control
states. Furthermore, the magnitude of the treatment effect grows over
time, with the point estimates trending upward throughout the
post-policy period. This suggests that the incentive to remove
high-risk, single-walled tanks was not a temporary shock but a
persistent and strengthening behavioral response to risk-based pricing.
Together, these dynamic results reinforce the causal interpretation of
the findings in Table~\ref{tbl-reg_closure}, demonstrating that the
introduction of risk-adjusted insurance led to an immediate and growing
increase in the removal of the riskiest assets from the market.

\subsection{Effect of Risk-Based Pricing on Reported
Leaks}\label{effect-of-risk-based-pricing-on-reported-leaks}

The preceding analysis demonstrates that risk-based insurance pricing
induced a significant behavioral response, accelerating the closure and
replacement of high-risk, single-walled tanks. The final and most
critical question is whether this behavioral change translated into a
tangible reduction in environmental harm. To assess this, I examine the
policy's impact on the frequency of reported leaks, the primary measure
of environmental incidents in this context.

The transition to risk-based pricing is expected to reduce leak
incidents through two primary channels, both of which were identified in
the previous section:

The Replacement Effect: By incentivizing the replacement of old,
single-walled tanks with newer, safer technology, the policy should
mechanically lower the aggregate leak rate over time. The Exit Effect:
The policy may also induce the permanent exit of the highest-risk or
least profitable facilities, further reducing the stock of leak-prone
tanks. To estimate the combined impact of these channels, I employ a
difference-in-differences specification analogous to the one used for
closure decisions. The model is estimated on the same sample of
single-walled tanks operating before 1999 to isolate the policy's effect
on the incumbent, high-risk cohort:

\[ 
\text{Reported Leak}_{i,t} = \beta \cdot \text{Texas}_{i} \times \text{Post1999}_{t} + \sum_{a=1}^{N} \gamma_a \cdot \mathbf{1}[\text{AgeBin}_a]_{i,t} + \alpha_i + \alpha_t + \epsilon_{i,t} 
\]

Here, the dependent variable, \(\text{Reported Leak}_{i,t}\), is a
binary indicator equal to one if facility \(i\) reported a leak in year
\(t\). The coefficient of interest, \(\beta\), captures the average
change in the probability of a reported leak for Texas facilities after
the policy transition, relative to control states. The model includes a
full set of facility (\(\alpha_i\)) and year (\(\alpha_t\)) fixed
effects to control for time-invariant facility characteristics and
common time shocks, respectively. It also includes flexible controls for
tank age (\(\mathbf{1}[\text{AgeBin}a]{i,t}\)) to account for the
natural lifecycle of leak risk. A negative and significant estimate for
\(\beta\) would provide strong evidence that risk-based insurance
pricing was successful in reducing environmental damages.

\begin{table}

\caption{\label{tbl-leak_results}\textbf{Policy effects on single-walled
facility reported leak rates}.}

\centering{

  \centering
  \begin{tabular}{lcc}
    \tabularnewline \midrule \midrule
    \textbf{Dependent Variable:} & \multicolumn{2}{c}{\textbf{Reported Leak  (0/1)}}\\
    \textbf{Model:}              & (1)                   & (2)\\
    \midrule
    Texas × Post-1999            & -0.0097$^{***}$       & -0.0124$^{***}$\\   
                        & (0.0030)              & (0.0032)\\
    \midrule
    Age Bin Controls             & No                    & Yes\\
    \midrule
    \textbf{Fixed-effects}\\
    Year FE                      & Yes                   & Yes\\
    Facility FE                  & Yes                   & Yes\\
  \midrule \midrule
    \multicolumn{3}{p{0.95\textwidth}}{\small \textit{Notes:} This table reports difference-in-differences estimates of the effect of Texas' 1999 transition from public to private UST insurance on reported leak rates for single-walled tank facilities operating before 1999. All models include facility and year fixed effects; model (2) adds controls for tank age. The results show that Texas facilities experienced a 0.97 to 1.24 percentage point reduction in the annual probability of a reported leak after the policy change. This suggests that risk-based insurance pricing was effective in reducing environmental incidents among high-risk facilities. Standard errors (in parentheses) are clustered at the state level. Significance: $^{***}p<0.01$, $^{**}p<0.05$, $^{*}p<0.1$.}
  \end{tabular}

}

\end{table}%

The results in Table~\ref{tbl-leak_results} show that the transition to
risk-based insurance pricing had a statistically significant and
negative impact on the probability of reported leaks among single-walled
tanks. The coefficient estimates of -0.0097 in column (1) and -0.0124 in
column (2) indicate that the policy reduced the annual probability of a
reported leak by approximately 0.97\% to 1.24\% relative to control
states. This reduction is economically meaningful, especially given the
baseline leak rate of around 1.13\% for single-walled tanks in the
pre-policy period. The inclusion of age bin controls in column (2) does
not materially change the magnitude or significance of the treatment
effect, suggesting that the policy's impact is robust to accounting for
differential leak rates across tank ages.

\section{Dynamic Discrete Choice
Model}\label{dynamic-discrete-choice-model}

In this section, I develop a dynamic structural model of the facility
owner's decision problem. The framework follows the optimal stopping
literature where agents weigh the immediate returns of operation against
the liability risks of aging capital Rust
(\citeproc{ref-rust_optimal_1987}{1987}).

\subsection{3.1 The Agent's Problem}\label{the-agents-problem}

I model the behavior of a risk-neutral facility owner operating a single
underground storage tank in an infinite horizon setting. Time is
discrete and indexed by \(t=1, 2, \dots, \infty\). In each period, the
agent observes the state of the facility, \(x_t\), and a vector of
idiosyncratic payoff shocks, \(\varepsilon_t\). Conditional on these
states, the agent chooses a binary action
\(d_t \in \{Maintain, Close\}\) to maximize the expected present
discounted value of future payoffs. The discount factor is denoted by
\(\beta \in (0, 1)\).

\subsection{State Space and
Transitions}\label{state-space-and-transitions}

The observable state vector \(x_t = (A_t, w_t, \rho_t)\) characterizes
the facility's physical and regulatory environment. The variable
\(A_t \in \{1, \dots, 9\}\) represents the age of the tank, discretized
into five-year bins, where \(A=9\) is an absorbing state for tanks older
than 40 years. The variable \(w_t \in \{Single, Double\}\) denotes the
tank's wall type, which is fixed over time. Finally,
\(\rho_t \in \{FF, RB\}\) indicates the insurance regime, where \(FF\)
represents the flat-fee public fund and \(RB\) represents the risk-based
private market.

The state transition process is governed by the facility's choice. If
the agent chooses to close the facility (\(d_t = Close\)), the process
terminates. If the agent chooses to maintain the facility
(\(d_t = Maintain\)), the tank ages stochastically. I define the
transition probability \(F(x_{t+1} | x_t, d_t)\) such that wall type and
regime are persistent, while age follows a stochastic process:

\[
A_{t+1} = \begin{cases} 
A_t + 1 & \text{with probability } \pi_{up}(A_t) \\
A_t & \text{with probability } 1 - \pi_{up}(A_t)
\end{cases}
\]

This stochastic aging process captures the uncertainty in asset
depreciation and allows the model to match the empirical duration of
tank active life.

\subsection{Flow Payoffs}\label{flow-payoffs}

The period payoff depends on the chosen action and the structural
parameters governing revenue, costs, and risk preferences. Payoffs are
normalized such that the net revenue from operation is the numeraire.

If the agent chooses to \textbf{Maintain}, the flow utility is specified
as:

\[
u(x_t, Maintain) = 1 + \gamma_{price} \cdot P(x_t) - \gamma_{risk} \cdot h(x_t) \cdot L(x_t)
\]

The first term, \(1\), represents the normalized annual net revenue. The
second term captures the disutility of insurance premiums, where
\(P(x_t)\) is the observed premium and \(\gamma_{price}\) is the price
sensitivity parameter. The third term represents the internalized cost
of environmental risk, where \(h(x_t)\) is the engineering hazard rate
of a leak, \(L(x_t)\) is the expected remediation cost, and
\(\gamma_{risk}\) is a scaling parameter. A value of
\(\gamma_{risk} < 1\) would imply moral hazard, where firms undervalue
the true expected liability.

If the agent chooses to \textbf{Close}, the facility exits the market
and the agent receives a terminal scrap value:

\[
u(x_t, Close) = \kappa
\]

The parameter \(\kappa\) captures the liquidation value of the land and
equipment, net of any decommissioning costs. This value is assumed to be
constant across states.

\subsection{The Value Function}\label{the-value-function}

The agent's optimization problem can be represented by the Bellman
equation. Assuming the idiosyncratic shocks \(\varepsilon_t\) follow a
Type I Extreme Value distribution with scale parameter \(\sigma\), the
ex-ante value function \(V(x_t)\) is given by:

\[
V(x_t) = \int \max_{d \in \{M, C\}} \left\{ v(x_t, d) + \varepsilon_t(d) \right\} dG(\varepsilon)
\]

where \(v(x_t, d)\) represents the conditional choice-specific value
function:

\[
v(x_t, d) = u(x_t, d) + \beta \sum_{x_{t+1}} V(x_{t+1}) F(x_{t+1} | x_t, d)
\]

This structure allows for the estimation of the structural parameters
\(\theta = (\kappa, \gamma_{price}, \gamma_{risk})\) using a Nested
Pseudo-Likelihood (NPL) estimator (Aguirregabiria \& Mira, 2007).

\subsection{Identification}\label{identification}

The identification of the structural parameters
\(\theta = (\kappa, \gamma_{price}, \gamma_{risk})\) relies on mapping
distinct sources of data variation to specific model primitives. While
the parameters are estimated jointly via the Nested Pseudo-Likelihood
algorithm, we can heuristically characterize the variation that
identifies each term.

The scrap value, \(\kappa\), is identified by the unconditional
probability of facility exit. In the optimal stopping framework,
\(\kappa\) acts as the reservation value; therefore, a higher average
exit rate across all states implies a higher relative scrap value,
holding flow profits constant. Identification requires that the baseline
exit rate is not driven solely by unobserved heterogeneity, which we
address through the inclusion of facility-level fixed effects in the
reduced-form steps.

The price sensitivity parameter, \(\gamma_{price}\), is identified by
the response of exit rates to cross-sectional variation in insurance
premiums. The transition from the flat-fee Public Fund (FF) to the
risk-based private market (RB) generates exogenous variation in
\(P(x_t)\) that is independent of the facility's engineering risk.
Specifically, under the FF regime, premiums are constant
(\(\frac{\partial P}{\partial A} = 0\)), whereas under the RB regime,
premiums rise with age (\(\frac{\partial P}{\partial A} > 0\)). The
covariance between these policy-induced price shocks and subsequent
closure decisions isolates the agent's marginal disutility of payment.

The risk internalization parameter, \(\gamma_{risk}\), is identified by
the gradient of exit rates with respect to the leak hazard \(h(x_t)\),
distinct from the premium channel. While premiums in the private market
are correlated with risk, the correlation is not perfect due to
deductible structures and coverage limits. If firms face liability costs
beyond the premium (e.g., deductibles or reputational harm), they will
exit high-risk states at a rate faster than predicted by premium costs
alone. This residual sensitivity of the exit rule to the engineering
hazard rate \(h(A_t)\) identifies \(\gamma_{risk}\).

\section{Discussion}\label{discussion}

The empirical results present a clear and compelling narrative about the
power of market-based environmental policy. The transition to
risk-adjusted insurance premiums in Texas prompted a significant and
rational response from the owners of the highest-risk USTs. Faced with
insurance costs that reflected the true environmental hazard of their
aging, single-walled tanks, owners accelerated their closure.
Critically, this did not lead to a mass exodus from the market. Instead,
the policy spurred investment in safer technology, as facilities were
more likely to replace their risky tanks rather than shut down entirely.
This finding directly addresses a common concern that stringent
environmental pricing will simply drive firms out of business,
demonstrating instead that it can be a powerful catalyst for technology
adoption.

This behavioral shift from retaining old, risky assets to investing in
new, safer ones translated directly into improved environmental
outcomes. The nearly one-percentage-point reduction in the annual leak
rate for this high-risk cohort is not only statistically significant but
also economically meaningful, representing a near-elimination of the
pre-policy baseline risk. This provides strong causal evidence that
aligning private insurance costs with public environmental risk can
effectively internalize externalities. The success of the Texas model
offers a clear lesson for the many states still relying on flat-fee
public funds, which mute price signals and subsidize the continued
operation of the riskiest facilities.

The findings underscore that when risk is observable and can be priced,
insurance markets can function as a powerful regulatory tool. The stark
difference in leak probabilities between single- and double-walled
tanks, which grows with age, provided a clear margin for insurers to
price on. By doing so, the private market created a persistent financial
incentive for risk mitigation that achieved tangible environmental
benefits, validating decades of economic theory on liability and
insurance design.

\section{Conclusion}\label{conclusion}

This paper provides the first facility-level causal evidence on the
impact of transitioning from a flat-fee public UST insurance fund to a
risk-rated private market. Using a difference-in-differences design that
leverages Texas's 1999 policy shift, the analysis demonstrates that
forcing firms to bear the expected cost of their environmental risk
profile has profound effects. The introduction of risk-based pricing led
to an immediate and sustained increase in the closure of high-risk,
single-walled tanks. This response was characterized not by market exit,
but by technology upgrading, as firms chose to replace their obsolete
tanks with safer alternatives. Ultimately, this behavioral change
resulted in a substantial and significant reduction in reported
environmental leaks.

The success of this price-based mechanism in reducing environmental risk
invites a crucial policy comparison. While this study highlights the
effectiveness of aligning market incentives with social costs, an
important avenue for future work is to contrast this approach with
traditional command-and-control regulations. For instance, comparing the
effects of risk-based pricing with the impact of vintage-differentiated
regulations (VDRs) or direct technology mandates could illuminate the
relative cost-effectiveness and efficiency of these different policy
instruments Shavell (\citeproc{ref-shavell_chapter_2007}{2007}). Such
analysis would provide regulators with a more complete understanding of
the optimal policy mix for managing the significant environmental
hazards posed by aging infrastructure in the U.S. and beyond.

\clearpage

\section{References}\label{references}

\phantomsection\label{refs}
\begin{CSLReferences}{1}{0}
\bibitem[\citeproctext]{ref-astswmo_archive}
ASTSWMO. 2025. {``Annual State Fund Survey Results --- Archive.''} 2025.
\url{https://astswmo.org/annual-state-fund-survey-results/}.

\bibitem[\citeproctext]{ref-boomhower_drilling_2019}
Boomhower, Judson. 2019. {``Drilling Like There's No Tomorrow:
Bankruptcy, Insurance, and Environmental Risk.''} \emph{American
Economic Review} 109 (2): 391--426.
\url{https://doi.org/10.1257/aer.20160346}.

\bibitem[\citeproctext]{ref-boyd_retroactive_1997}
Boyd, James, and Howard Kunreuther. 1997. {``Retroactive Liability or
the Public Purse?''} \emph{Journal of Regulatory Economics} 11 (1):
79--90. \url{https://doi.org/10.1023/A:1007954314303}.

\bibitem[\citeproctext]{ref-brown_how_2014}
Brown, Jason, Mark Duggan, Ilyana Kuziemko, and William Woolston. 2014.
{``How Does Risk Selection Respond to Risk Adjustment? New Evidence from
the Medicare Advantage Program.''} \emph{American Economic Review} 104
(10): 3335--64. \url{https://doi.org/10.1257/aer.104.10.3335}.

\bibitem[\citeproctext]{ref-csi2021}
Containment Solutions, Inc. 2021. {``PG 11000AC Petroleum Tank Price
List, Effective January 1 2021.''} Manufacturer price catalogue.
\url{https://www.containmentsolutions.com}.

\bibitem[\citeproctext]{ref-einav_io_2021}
Einav, Liran, Amy Finkelstein, and Neale Mahoney. 2021. {``The {IO} of
Selection Markets.''} \emph{{NBER} Working Papers}, July.
\url{https://ideas.repec.org//p/nbr/nberwo/29039.html}.

\bibitem[\citeproctext]{ref-einav_contract_2012}
Einav, Liran, Mark Jenkins, and Jonathan Levin. 2012. {``Contract
Pricing in Consumer Credit Markets.''} \emph{Econometrica} 80 (4):
1387--1432. \url{https://doi.org/10.3982/ECTA7677}.

\bibitem[\citeproctext]{ref-einav_impact_2013}
---------. 2013. {``The Impact of Credit Scoring on Consumer Lending.''}
\emph{The {RAND} Journal of Economics} 44 (2): 249--74.
\url{https://doi.org/10.1111/1756-2171.12019}.

\bibitem[\citeproctext]{ref-epa_underground_2024}
EPA. 2024. {``Underground Storage Tank Program Facts.''}
\emph{Underground Storage Tank Program Facts}, May, 2.
\url{https://www.epa.gov/system/files/documents/2024-05/ust-programfacts-may2024.pdf}.

\bibitem[\citeproctext]{ref-gruenspecht_differentiated_1982}
Gruenspecht, Howard K. 1982. {``Differentiated Regulation: The Case of
Auto Emissions Standards.''} \emph{The American Economic Review} 72 (2):
328--31. \url{https://www.jstor.org/stable/1802352}.

\bibitem[\citeproctext]{ref-guignet_what_2013}
Guignet, Dennis. 2013. {``What Do Property Values Really Tell Us? A
Hedonic Study of Underground Storage Tanks.''} \emph{Land Economics} 89
(2): 211--26. \url{https://doi.org/10.3368/le.89.2.211}.

\bibitem[\citeproctext]{ref-guignet_sell_2014}
---------. 2014. {``To Sell or Not to Sell: The Impacts of Pollution on
Home Transactions.''} \emph{National Center for Environmental
Economics-{NCEE} Working Papers}, January.
\url{https://ideas.repec.org//p/ags/nceewp/280917.html}.

\bibitem[\citeproctext]{ref-guignet_contamination_2018}
Guignet, Dennis, Robin Jenkins, Matthew Ranson, and Patrick J. Walsh.
2018. {``Contamination and Incomplete Information: Bounding Implicit
Prices Using High-Profile Leaks.''} \emph{Journal of Environmental
Economics and Management} 88 (March): 259--82.
\url{https://doi.org/10.1016/j.jeem.2017.12.003}.

\bibitem[\citeproctext]{ref-guignet_impacts_2018}
Guignet, Dennis, and Adan Martinez-Cruz. 2018. {``The Impacts of
Underground Petroleum Releases on a Homeowner's Decision to Sell: A
Difference-in-Differences Approach.''} \emph{Regional Science and Urban
Economics} 69: 11--24.
\url{https://doi.org/10.1016/j.regsciurbeco.2017.12.006}.

\bibitem[\citeproctext]{ref-gao1987}
{``Hazardous Materials: Upgrading of Underground Storage Tanks Can Be
Improved to Avoid Costly Cleanups.''} 1992. NSIAD-92-117. U.S.
Government Accountability Office.
\url{https://www.gao.gov/assets/nsiad-92-117.pdf}.

\bibitem[\citeproctext]{ref-de_janvry_supply-_2010}
Janvry, Alain de, Craig McIntosh, and Elisabeth Sadoulet. 2010. {``The
Supply- and Demand-Side Impacts of Credit Market Information.''}
\emph{Journal of Development Economics} 93 (2): 173--88.
\url{https://doi.org/10.1016/j.jdeveco.2009.09.008}.

\bibitem[\citeproctext]{ref-liberman_equilibrium_2018}
Liberman, Andres, Christopher Neilson, Luis Opazo, and Seth Zimmerman.
2018. {``The Equilibrium Effects of Information Deletion: Evidence from
Consumer Credit Markets.''} \emph{{NBER} Working Papers}, September.
\url{https://ideas.repec.org//p/nbr/nberwo/25097.html}.

\bibitem[\citeproctext]{ref-marcus_going_2021}
Marcus, Michelle. 2021. {``Going Beneath the Surface: Petroleum
Pollution, Regulation, and Health.''} \emph{American Economic Journal:
Applied Economics} 13 (1): 72--104.
\url{https://doi.org/10.1257/app.20190130}.

\bibitem[\citeproctext]{ref-mcwilliams_new_2012}
McWilliams, J. Michael, John Hsu, and Joseph P. Newhouse. 2012. {``New
Risk-Adjustment System Was Associated with Reduced Favorable Selection
in Medicare Advantage.''} \emph{Health Affairs (Project Hope)} 31 (12):
2630--40. \url{https://doi.org/10.1377/hlthaff.2011.1344}.

\bibitem[\citeproctext]{ref-nelson_private_2025}
Nelson, Scott T. 2025. {``Private Information and Price Regulation in
the {US} Credit Card Market.''} \emph{Econometrica} Forthcoming
(February).

\bibitem[\citeproctext]{ref-polinsky_strict_1980}
Polinsky, A. Mitchell. 1980. {``Strict Liability Vs. Negligence in a
Market Setting.''} \emph{The American Economic Review} 70 (2): 363--67.
\url{https://www.jstor.org/stable/1815499}.

\bibitem[\citeproctext]{ref-rust_optimal_1987}
Rust, John. 1987. {``Optimal Replacement of {GMC} Bus Engines: An
Empirical Model of Harold Zurcher.''} \emph{Econometrica} 55 (5):
999--1033. \url{https://doi.org/10.2307/1911259}.

\bibitem[\citeproctext]{ref-shavell_liability_1982}
Shavell, Steven. 1982. {``On Liability and Insurance.''} \emph{The Bell
Journal of Economics} 13 (1): 120--32.
\url{https://doi.org/10.2307/3003434}.

\bibitem[\citeproctext]{ref-shavell_model_1984}
---------. 1984a. {``A Model of the Optimal Use of Liability and Safety
Regulation.''} \emph{The {RAND} Journal of Economics} 15 (2): 271--80.
\url{https://doi.org/10.2307/2555680}.

\bibitem[\citeproctext]{ref-shavell_liability_1984}
---------. 1984b. {``Liability for Harm Versus Regulation of Safety.''}
\emph{The Journal of Legal Studies} 13 (2): 357--74.
\url{https://www.jstor.org/stable/724240}.

\bibitem[\citeproctext]{ref-shavell_chapter_2007}
---------. 2007. {``Chapter 2 Liability for Accidents.''} In
\emph{Handbook of Law and Economics}, 1:139--82. Elsevier.
\url{https://doi.org/10.1016/S1574-0730(07)01002-X}.

\bibitem[\citeproctext]{ref-walsh_contaminated_2017}
Walsh, Patrick, and Preston Mui. 2017. {``Contaminated Sites and
Information in Hedonic Models: An Analysis of a {NJ} Property Disclosure
Law.''} \emph{Resource and Energy Economics} 50 (November): 1--14.
\url{https://doi.org/10.1016/j.reseneeco.2017.06.005}.

\bibitem[\citeproctext]{ref-xerxes2018}
Xerxes Corporation. 2018. {``Xerxes Underground Storage Tanks and
Accessories Price List.''} Manufacturer price catalogue.
\url{https://www.xerxes.com}.

\bibitem[\citeproctext]{ref-yin_environmental_2007}
Yin, Haitao, Howard C. Kunreuther, and Matthew W. White. 2007. {``Do
Environmental Regulations Cause Firms to Exit the Market? Evidence from
Underground Storage Tank ({UST}) Regulations.''} \emph{{SSRN} Electronic
Journal}. \url{https://doi.org/10.2139/ssrn.1265611}.

\bibitem[\citeproctext]{ref-yin_risk-based_2011}
Yin, Haitao, Howard Kunreuther, and Matthew W. White. 2011.
{``Risk-Based Pricing and Risk-Reducing Effort: Does the Private
Insurance Market Reduce Environmental Accidents?''} \emph{Journal of Law
and Economics} 54 (2): 325--63.
\url{https://ideas.repec.org//a/ucp/jlawec/doi10.1086-655804.html}.

\bibitem[\citeproctext]{ref-yin_can_2011}
Yin, Haitao, Alex Pfaff, and Howard Kunreuther. 2011. {``Can
Environmental Insurance Succeed Where Other Strategies Fail? The Case of
Underground Storage Tanks.''} \emph{Risk Analysis} 31 (1): 12--24.
\url{https://doi.org/10.1111/j.1539-6924.2010.01479.x}.

\bibitem[\citeproctext]{ref-zabel_hedonic_2012}
Zabel, Jeffrey E., and Dennis Guignet. 2012. {``A Hedonic Analysis of
the Impact of {LUST} Sites on House Prices.''} \emph{Resource and Energy
Economics} 34 (4): 549--64.
\url{https://doi.org/10.1016/j.reseneeco.2012.05.006}.

\end{CSLReferences}



\end{document}
