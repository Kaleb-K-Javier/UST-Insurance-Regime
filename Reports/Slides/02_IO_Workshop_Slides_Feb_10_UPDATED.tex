\documentclass[aspectratio=169,]{beamer}

% -------------------------------------------------------------------------
% 1. REQUIRED PACKAGES (The "Batteries" Quarto usually includes)
% -------------------------------------------------------------------------
\usepackage{graphicx}
\usepackage{booktabs}
\usepackage{longtable}
\usepackage{hyperref}
\usepackage{array}      % <--- ADD THIS LINE HERE
\usepackage[most]{tcolorbox} % REQUIRED for Quarto Callouts
\usepackage{fontawesome5}    % REQUIRED for Callout Icons
\usepackage{framed}          % REQUIRED for standard blocks
\usepackage{tikz}
\usetikzlibrary{arrows.meta}

% Fix for Pandoc's "tightlist"
\providecommand{\tightlist}{%
  \setlength{\itemsep}{0pt}\setlength{\parskip}{0pt}}

% -------------------------------------------------------------------------
% 2. CUSTOM DESIGN PACKAGES
% -------------------------------------------------------------------------
\usepackage{tikz}
\usepackage{calc}
\usepackage{xcolor}
\usepackage{xparse} 
\usepackage{etoolbox} 
\usepackage[default]{sourcesanspro} 
\usetikzlibrary{calc, positioning, shapes.geometric}

% -------------------------------------------------------------------------
% DEFINE QUARTO CALLOUT COLORS (REQUIRED)
% -------------------------------------------------------------------------
% We map these to your "Elegant Dark Mode" aesthetic.
% Note: Quarto expects these EXACT names.

% 1. Note (Blue)
\definecolor{quarto-callout-note-color}{HTML}{3498DB}        % Icon/Title Color
\definecolor{quarto-callout-note-color-frame}{HTML}{2980B9}  % Border Color

% 2. Tip (Green)
\definecolor{quarto-callout-tip-color}{HTML}{27AE60}
\definecolor{quarto-callout-tip-color-frame}{HTML}{2ECC71}

% 3. Important (Red/Crimson)
\definecolor{quarto-callout-important-color}{HTML}{C0392B}
\definecolor{quarto-callout-important-color-frame}{HTML}{E74C3C}

% 4. Warning (Orange)
\definecolor{quarto-callout-warning-color}{HTML}{F39C12}
\definecolor{quarto-callout-warning-color-frame}{HTML}{E67E22}

% 5. Caution (Yellow/Gold)
\definecolor{quarto-callout-caution-color}{HTML}{F1C40F}
\definecolor{quarto-callout-caution-color-frame}{HTML}{D4AC0D}

% Add this right before \begin{document}
\newcommand{\Est}[1]{\textcolor{quarto-callout-important-color}{\mathbf{#1}}}


% -------------------------------------------------------------------------
% NEW: COLOR THEME COMMANDS (Robust Single-Slide Overrides)
% -------------------------------------------------------------------------

% 1. Light Mode Command
\newcommand{\LightMode}{
  % 1. Draw White Background (TikZ Overlay)
  \begin{tikzpicture}[remember picture, overlay]
    \fill[white] (current page.south west) rectangle (current page.north east);
  \end{tikzpicture}
  % 2. Force Text Colors to Black
  \setbeamercolor{normal text}{fg=black}
  \setbeamercolor{frametitle}{fg=black}
  \setbeamercolor{itemize item}{fg=black}
  \setbeamercolor{structure}{fg=black}
  \usebeamercolor[fg]{normal text}
}

% 2. Black Mode Command
\newcommand{\BlackMode}{
  \begin{tikzpicture}[remember picture, overlay]
    \fill[black] (current page.south west) rectangle (current page.north east);
  \end{tikzpicture}
  \setbeamercolor{normal text}{fg=white}
  \setbeamercolor{frametitle}{fg=white}
  \setbeamercolor{itemize item}{fg=white}
  \setbeamercolor{structure}{fg=white}
  \usebeamercolor[fg]{normal text}
}

% 3. Brand Mode Command
\newcommand{\BrandMode}{
  \begin{tikzpicture}[remember picture, overlay]
    \definecolor{tempBrand}{HTML}{7D181E}
    \fill[tempBrand] (current page.south west) rectangle (current page.north east);
  \end{tikzpicture}
  \setbeamercolor{normal text}{fg=white}
  \setbeamercolor{frametitle}{fg=white}
  \setbeamercolor{itemize item}{fg=white}
  \setbeamercolor{structure}{fg=white}
  \usebeamercolor[fg]{normal text}
}

% -------------------------------------------------------------------------
% 4. UC BERKELEY THEME (Modern Matte) - FIXED SCOPE
% -------------------------------------------------------------------------

% 1. DEFINE COLORS GLOBALLY (Required for assumptionbox and modes)
\definecolor{BerkeleyBlue}{HTML}{003262}
\definecolor{CaliforniaGold}{HTML}{FDB515}
\definecolor{BerkeleyMatte}{HTML}{0D2D52} 
\definecolor{SoftWhite}{HTML}{F8F9FA}

% 4. UC Berkeley Theme (Modern Matte)
\newcommand{\BerkeleyMode}{
  % 2. Draw Background AND Title
  \begin{tikzpicture}[remember picture, overlay]
    % Background Fill (Covers old title)
    \fill[BerkeleyMatte] (current page.south west) rectangle (current page.north east);
    % Gold Accent Strip
    \fill[CaliforniaGold] (current page.south west) rectangle ($(current page.north west)+(0.15cm,0)$);
    % REDRAW TITLE ON TOP
    \node[anchor=west, white, font=\Large\bfseries] at ($(current page.north west)+(1cm,-1cm)$) {\insertframetitle};
  \end{tikzpicture}
  
  % 3. Set Text Colors
  \setbeamercolor{normal text}{fg=SoftWhite}
  \setbeamercolor{frametitle}{fg=BerkeleyMatte} % Hide original title
  \setbeamercolor{itemize item}{fg=CaliforniaGold} 
  \setbeamercolor{structure}{fg=CaliforniaGold}
  
  \usebeamercolor[fg]{normal text}
}

% 5. UC Berkeley Theme (Academic Light)
\newcommand{\BerkeleyLightMode}{
  % (Color definitions removed here as they are now global)
  
  % 2. Draw Background AND Title
  \begin{tikzpicture}[remember picture, overlay]
    % Background Fill
    \fill[white] (current page.south west) rectangle (current page.north east);
    % Blue Header Bar
    \fill[BerkeleyBlue] (current page.north west) rectangle ($(current page.north east)+(0,-1.4cm)$);
    % Gold Line
    \fill[CaliforniaGold] ($(current page.north west)+(0,-1.4cm)$) rectangle ($(current page.north east)+(0,-1.45cm)$);
    % REDRAW TITLE ON TOP
    \node[anchor=west, white, font=\Large\bfseries] at ($(current page.north west)+(1cm,-0.7cm)$) {\insertframetitle};
  \end{tikzpicture}
  
  % 3. Set Text Colors
  \setbeamercolor{normal text}{fg=black}
  \setbeamercolor{frametitle}{fg=white} 
  \setbeamercolor{itemize item}{fg=BerkeleyBlue}
  \setbeamercolor{structure}{fg=BerkeleyBlue}
  
  \usebeamercolor[fg]{normal text}
}

% -------------------------------------------------------------------------
% NEW: ACADEMIC POWER LAYOUTS
% -------------------------------------------------------------------------

% 1. Statement Layout (Big Impact Transitions)
% Usage: ::: {.layout-statement} Content :::
\newenvironment{layout-statement}{
  \vspace*{\fill}
  \centering
  \Huge \bfseries
  \setbeamercolor{normal text}{fg=CaliforniaGold} % Use Gold for impact
  \usebeamercolor[fg]{normal text}
}{
  \vspace*{\fill}
}

% 2. Dense Data Layout (Regression Tables)
% Usage: ::: {.layout-dense} Table :::
\newenvironment{layout-dense}{
  \vspace*{\fill}
  \tiny                  % Force tiny font
  \setlength\tabcolsep{2pt} % Tighten table columns
  \centering
}{
  \vspace*{\fill}
}

% 3. Split Slide Macro (Vertical Centering)
% Usage: \SplitSlide{Left Content}{Right Content}
\newcommand{\SplitSlide}[2]{
  \begin{columns}[c] % 'c' option forces vertical centering
    \begin{column}{0.48\textwidth}
      #1
    \end{column}
    \begin{column}{0.48\textwidth}
      \centering
      #2
    \end{column}
  \end{columns}
}

% -------------------------------------------------------------------------
% 3. DYNAMIC BACKGROUND & CONTRAST LOGIC
% -------------------------------------------------------------------------
  \definecolor{mainbg}{HTML}{0D2D52}

\setbeamercolor{background canvas}{bg=mainbg}

% Logic to calculate luminance and set text color automatically
\makeatletter
\newcommand{\AutoSetContrast}{
  \extractcolorspec{mainbg}{\mainbgspec}
  \expandafter\convertcolorspec\mainbgspec{rgb}\mainbgrgb
  \def\calculatebrightness##1,##2,##3;{
    \pgfmathparse{0.2126*##1 + 0.7152*##2 + 0.0722*##3}
  }
  \expandafter\calculatebrightness\mainbgrgb;
  \pgfmathparse{\pgfmathresult > 0.5 ? 1 : 0}
  \ifnum\pgfmathresult=1
    \definecolor{maintext}{HTML}{222222}
    \definecolor{dimtext}{HTML}{555555}
  \else
    \definecolor{maintext}{HTML}{FFFFFF}
    \definecolor{dimtext}{HTML}{CCCCCC}
  \fi
}
\makeatother

\AutoSetContrast

\setbeamercolor{normal text}{fg=maintext}
\setbeamercolor{frametitle}{fg=maintext}
\setbeamercolor{title}{fg=maintext}
\setbeamercolor{structure}{fg=maintext} 

% -------------------------------------------------------------------------
% 4. FOOTER CUSTOMIZATION
% -------------------------------------------------------------------------
\setbeamercolor{footline}{fg=dimtext}
\setbeamerfont{footline}{size=\tiny}

\newtoggle{hidesection}
\togglefalse{hidesection}

\setbeamertemplate{footline}{
  \begin{beamercolorbox}[wd=\paperwidth, ht=2.5ex, dp=1.125ex, leftskip=.3cm, rightskip=.3cm]{footline}
    \iftoggle{hidesection}{}{
      \insertsectionhead
    }
    \hfill
    \insertframenumber
  \end{beamercolorbox}
}
\setbeamertemplate{navigation symbols}{}

% -------------------------------------------------------------------------
% 5. CUSTOM COMMANDS & ENVIRONMENTS (FIXED FOR STABILITY)
% -------------------------------------------------------------------------
\NewDocumentCommand{\navbutton}{ O{} O{fill=gray} m }{
  \hyperlink{#1}{
    \tikz[baseline=(node.base)]{
      \node(node)[
        anchor=base,
        rounded corners=3pt,
        inner sep=5pt,
        text=white,
        font=\small\bfseries,
        #2
      ]{#3};
    }
  }
}

\newenvironment{layout-math}{
  \vspace*{\fill}
  \centering
  \begin{minipage}{\linewidth}
  \centering
}{
  \end{minipage}
  \vspace*{\fill}
}

% SAFE Layout for Tables (Removed risky \makebox)
\newenvironment{layout-table}{
  \vspace*{\fill}
  \begin{center}
}{
  \end{center}
  \vspace*{\fill}
}

% SAFE Layout for Full Figures (Uses LRBOX to capture content first)
\newsavebox{\fullfigbox}
\newenvironment{layout-full-figure}{
  \begin{lrbox}{\fullfigbox}
    \begin{minipage}{\paperwidth}
      \centering
}{
    \end{minipage}
  \end{lrbox}
  \begin{tikzpicture}[remember picture, overlay]
    \node[anchor=center, inner sep=0pt] at (current page.center) {
      \usebox{\fullfigbox}
    };
  \end{tikzpicture}
}

\newcommand{\hidefootersection}{\global\toggletrue{hidesection}}
\newcommand{\showfootersection}{\global\togglefalse{hidesection}}


% -------------------------------------------------------------------------
% CUSTOM BERKELEY CALLOUT BOX
% -------------------------------------------------------------------------
\newtcolorbox{assumptionbox}[1]{
  enhanced,
  colback=SoftWhite,       % White-ish background
  colframe=CaliforniaGold, % Gold Border
  coltitle=SoftWhite,      % White Title Text
  colbacktitle=BerkeleyMatte, % Berkeley Blue Title Background
  title={#1},
  fonttitle=\bfseries\small,
  arc=2pt,                 % Slight rounded corners
  boxrule=1pt,             % Thin elegant border
  left=5pt, right=5pt, top=5pt, bottom=5pt, % Padding
  titlerule=0pt,
  width=\linewidth
}

% -------------------------------------------------------------------------
% 6. DOCUMENT BODY
% -------------------------------------------------------------------------
\title{Internalizing Environmental Risk: Insurance Design and Firm
Behavior in Hazardous Industries}
\author{Kaleb K. Javier}

\begin{document}

\frame{\titlepage}

\begin{frame}{Goals for Today's Talk}
\phantomsection\label{goals-for-todays-talk}
\BerkeleyLightMode

\textbf{1. Toy Model: Is the Focus Right?}

\begin{itemize}
\tightlist
\item
  Does the simple framework clearly capture the behavior I can measure
  (tank closures)?
\item
  Thoughts on how to sharpen the premium \(\rightarrow\) closure
  mechanism?
\end{itemize}

\textbf{2. Structural Model: What's Missing?}

\begin{itemize}
\tightlist
\item
  Is the DDC setup complete, or are key state variables/parameters
  omitted?
\item
  \textbf{Specific question:} How do I incorporate Marcus (2021) health
  externalities without estimating damages myself? Just use their
  estimates as inputs?
\item
  Best ways to push this forward?
\end{itemize}
\end{frame}

\begin{frame}{Motivation: Mispricing Environmental Risk}
\phantomsection\label{motivation-mispricing-environmental-risk}
\BerkeleyLightMode

\textbf{The Environmental Problem}

\begin{itemize}
\tightlist
\item
  USTs are the leading source of groundwater contamination (EPA, 2024);
  median cleanup cost \(\sim \$150{,}000\) and mean costs
  \(\sim \$420{,}000\)
\end{itemize}

\textbf{The Policy Paradox}

\begin{itemize}
\tightlist
\item
  Federal law mandates strict liability insurance to internalize
  remediation costs.
\item
  Strict liability works only when \(P_i \propto Risk_i\) (Shavell,
  1982), but uniform premiums \(\bar{P}\) break this link.
\end{itemize}

\textbf{The Technological Friction}

\begin{itemize}
\tightlist
\item
  Single-walled tanks (riskiest types) persist: 73\% of Texas stock as
  of 1999 and 49\% in 2018, 75\% of Control states in 1999 and 63\% in
  2018
\item
  Double-walled replacement costs 60--90\% more (GAO, 1987).
\end{itemize}

\textbf{The ``Texas Experiment'' (1999)}

\begin{itemize}
\tightlist
\item
  \textbf{Shock:} Petroleum Storage Tank (PST) Fund sunset forces 100\%
  of owners into risk-rated private market.
\item
  \textbf{Prediction:} Risk-based pricing should induce tank closures of
  high-risk facilities.
\end{itemize}
\end{frame}

\begin{frame}{Research Question}
\phantomsection\label{research-question}
\BerkeleyLightMode

\textbf{Does replacing uniform-premium public insurance with risk-based
private coverage induce firms to internalize environmental damages?}

\textbf{Two Behavioral Margins}

\begin{enumerate}
\tightlist
\item
  \textbf{Tank Closures} (Optimal Stopping): Accelerated closure of
  aging, high-risk tanks
\item
  \textbf{Outcome}: Net change in Leaking Underground Storage Tank
  (LUST) reporting events
\end{enumerate}

\textbf{Modeling Strategy}

\begin{itemize}
\tightlist
\item
  Dynamic discrete choice (DDC) framework (Rust, 1987): Owner chooses
  \(\{Close, Continue\}\) each period. Insurance regime shift changes
  effective cost function.
\end{itemize}
\end{frame}

\begin{frame}{Contribution}
\phantomsection\label{contribution}
\BerkeleyLightMode

\textbf{1. Identification: Correcting OVB in Prior Literature}

\begin{itemize}
\tightlist
\item
  \textbf{Yin et al.~(2011):} State-level aggregates \(\rightarrow\)
  composition bias, regional shocks.
\item
  \textbf{This Paper:} First facility-level panel with tank-specific
  covariates.
\item
  \textbf{Gain:} Facility fixed effects isolate policy response from
  heterogeneity (location, geology).
\end{itemize}

\textbf{2. Mechanism: Structural Decomposition}

\begin{itemize}
\tightlist
\item
  Tests whether insurance pricing affects ex-ante moral hazard (Einav et
  al., 2021).
\item
  DDC model estimates optimal stopping rule \(\rightarrow\) enables
  counterfactual policy analysis.
\end{itemize}

\textbf{3. Policy: Pricing Design as Dynamic Lever}

\begin{itemize}
\tightlist
\item
  Beyond bonding mandates (Boomhower, 2019): pricing structure within
  assurance matters.
\item
  Links closure behavior to LUST outcomes and health externalities
  (Marcus, 2021; Kellogg \& Reguant, 2021).
\end{itemize}
\end{frame}

\begin{frame}{Underground Storage Tank (UST) Regulatory Context}
\phantomsection\label{underground-storage-tank-ust-regulatory-context}
\BerkeleyLightMode

\begin{center}
\begin{tikzpicture}[
  x=0.4cm,
  federal/.style={above, align=center, font=\small, text width=4cm},
  texas/.style={below, align=center, font=\small, text=BerkeleyBlue, text width=5cm},
  yearlab/.style={font=\small\bfseries}
]

% --- Axis ---
\draw[thick, -{Stealth[length=3mm]}] (-1,0) -- (27,0);

% --- Ticks ---
\foreach \x in {0, 8, 12, 22}{
  \draw (\x, 0.15) -- (\x, -0.15);
}

% --- Year labels ---
\node[yearlab, below] at (0, -0.3) {1988};
\node[yearlab, below] at (8, -0.3) {1998};
\node[yearlab, below] at (12, -0.3) {1999};
\node[yearlab, below] at (22, -0.3) {2005};

% --- Federal events (above line) ---
\draw[gray, thick] (0, 0.15) -- (0, 1.0);
\node[federal] at (0, 1.2)
  {RCRA Subtitle I\\[-2pt] {\scriptsize Federal UST\\program begins}};

\draw[gray, thick] (8, 0.15) -- (8, 1.0);
\node[federal] at (8, 1.2)
  {Retrofit Deadline\\[-2pt] {\scriptsize Upgrade or close\\non-compliant tanks}};

\draw[gray, thick] (22, 0.15) -- (22, 1.0);
\node[federal] at (22, 1.2)
  {Energy Policy Act\\[-2pt] {\scriptsize 3-yr inspections,\\secondary containment}};

% --- Texas event (below line) ---
\draw[BerkeleyBlue, very thick] (12, -0.15) -- (12, -1.2);
\node[texas] at (12, -1.5)
  {\textbf{Texas: Public Fund Closes}\\[-2pt] {\scriptsize Owners forced into\\risk-based private market ($P_{it}$)}};

% --- Bracket labels ---
\node[above, font=\scriptsize\itshape, gray] at (8, 2.5) {\textit{All States}};
\node[below, font=\scriptsize\itshape, BerkeleyBlue] at (12, -2.7) {\textit{Texas Only}};

\end{tikzpicture}
\end{center}

\vfill

All federal technical mandates apply identically to Texas and control
states. Only the insurance regime diverges.
\end{frame}

\begin{frame}{}
\phantomsection\label{section}
\BerkeleyMode

\begin{center}
\Huge \textbf{A Dynamic Model of Tank Closure}
\end{center}
\end{frame}

\begin{frame}{The Agent's Decision}
\phantomsection\label{the-agents-decision}
\BerkeleyLightMode

\textbf{The Choice}

Each period \(t\), the facility chooses:
\(d_t \in \{\text{Maintain}, \text{Close}\}\)

\textbf{The Trade-off}

\begin{itemize}
\tightlist
\item
  \textbf{Maintain:} Earn net revenue, pay insurance premiums.
\item
  \textbf{Close:} Exit market and recover scrap value \(\kappa\)
  (liquidity of land).
\end{itemize}

\textbf{Flow Utility (Maintain)} \[
u(a) = \underbrace{R}_{\text{Net Revenue}} - \underbrace{C_j(a)}_{\text{Insurance Premium}}
\]

\begin{assumptionbox}{Simplifying Assumption: Full Coverage}
Firms face environmental risk solely as an \textit{ex-ante} cost (premium), not an \textit{ex-post} shock (deductible).
\end{assumptionbox}
\end{frame}

\begin{frame}{The Dynamic Problem}
\phantomsection\label{the-dynamic-problem}
\BerkeleyLightMode

\textbf{The Bellman Equation}

\[
V(a) = \max \Bigg\{ \underbrace{u(a) + \beta \mathbb{E}[V(a') \mid a]}_{\text{Value of Maintaining}}, \quad \underbrace{\kappa}_{\text{Scrap Value}} \Bigg\}
\]

\textbf{The Continuation Value (\(V_{cont}\))} The value of keeping the
tank active today to preserve the option of operating tomorrow: \[
V_{cont}(a) = u(a) + \beta \mathbb{E}[V(a') \mid a]
\]

\textbf{Optimal Stopping Rule} \[
\text{Close if } \quad V_{cont}(a) < \kappa
\]
\end{frame}

\begin{frame}{The Policy Lever: Insurance Regimes}
\phantomsection\label{the-policy-lever-insurance-regimes}
\BerkeleyLightMode

\textbf{1. Uniform Premium Pooling (\(F\))}

\begin{itemize}
\tightlist
\item
  \textbf{Premium:} Constant \(\bar{P}\).
\item
  \textbf{Incentive:} Zero marginal cost of aging. \[
  \frac{\partial C}{\partial \text{Age}} = 0
  \]
\end{itemize}

\textbf{2. Risk-Based Pricing (\(RB\))}

\begin{itemize}
\tightlist
\item
  \textbf{Premium:} Rises with leak hazard \(h(a)\), the instantaneous
  probability of a leak, strictly increasing in age.
\item
  \textbf{Incentive:} Positive marginal cost of aging. \[
  \frac{\partial C}{\partial \text{Age}} > 0
  \]
\end{itemize}
\end{frame}

\begin{frame}{What's the First Best?}
\phantomsection\label{whats-the-first-best}
\BerkeleyLightMode

\textbf{Social Planner's Objective}

Internalize remediation costs (\(L\)) \emph{and} health externalities
(\(E\)):

\[
u_{SOC}(a) = R - h(a) \cdot (\underbrace{L}_{\text{Remediation}} + \underbrace{E}_{\text{Health}})
\]

\textbf{Why Risk-Based Pricing Falls Short}

\begin{itemize}
\tightlist
\item
  Actuarial premiums reflect \emph{insurer claims data} \(\rightarrow\)
  only remediation costs (\(L\)).
\item
  Health damages (\(E\)) generate no third-party claims: contamination
  is unobservable (Marcus, 2021).
\item
  \(\Rightarrow\) Even fair risk-based pricing leaves \(E\)
  externalized.
\end{itemize}

\textbf{Implication}

\[
V_{RB}(a) > V_{SOC}(a) \quad \text{because} \quad h(a) \cdot L < h(a)(L + E)
\]

Risk-based pricing induces \emph{some} closures, but not enough.
\end{frame}

\begin{frame}{The Three Facts to Keep in Mind}
\phantomsection\label{the-three-facts-to-keep-in-mind}
\BerkeleyLightMode

\textbf{1. Asset Depreciation}

Rising leak risk causes tank value to strictly decrease in age
(\(V'(a) < 0\)).

\begin{itemize}
\tightlist
\item
  \emph{Economic Friction:} The regime determines if private incentives
  align with this social decay.
\end{itemize}

\textbf{2. Inefficient Retention (Uniform Premium)}

\begin{itemize}
\tightlist
\item
  \textbf{Mechanism:} Static costs (\(\bar{P}\)) \(\implies\) Zero
  marginal cost of aging.
\item
  \textbf{Result:} Firm ignores risk; retains asset longest
  (\(a^*_{UP}\)).
\end{itemize}

\textbf{3. Partial Correction (Risk-Based)}

\begin{itemize}
\tightlist
\item
  \textbf{Mechanism:} Rising premiums (\(h(a)L\)) \(\implies\) Positive
  marginal cost of aging.
\item
  \textbf{Result:} Accelerated exit (\(a^*_{RB}\)), yet strictly \(>\)
  \(a^*_{SOC}\) due to unpriced health costs.
\end{itemize}

\textbf{The Ordering}

\[
a^*_{SOC} < a^*_{RB} < a^*_{UP}
\]
\end{frame}

\begin{frame}{}
\phantomsection\label{section-1}
\LightMode

\includegraphics[width=0.95\textwidth,height=0.95\textheight]{../../Output/Figures/Slide_Fig_2_SOC_Exit.pdf}
\end{frame}

\begin{frame}{}
\phantomsection\label{section-2}
\LightMode

\includegraphics[width=0.95\textwidth,height=\textheight]{../../Output/Figures/Slide_Fig_3_SOC_RB_Comparison.pdf}
\end{frame}

\begin{frame}{}
\phantomsection\label{section-3}
\LightMode

\includegraphics[width=0.95\textwidth,height=\textheight]{../../Output/Figures/Slide_Fig_4_RB_DWL.pdf}
\end{frame}

\begin{frame}{}
\phantomsection\label{section-4}
\LightMode

\includegraphics[width=0.95\textwidth,height=\textheight]{../../Output/Figures/Slide_Fig_5_SOC_FF_Comparison.pdf}
\end{frame}

\begin{frame}{}
\phantomsection\label{section-5}
\LightMode

\includegraphics[width=0.95\textwidth,height=\textheight]{../../Output/Figures/Slide_Fig_6_FF_Total_DWL.pdf}
\end{frame}

\begin{frame}{}
\phantomsection\label{section-6}
\LightMode

\includegraphics[width=0.95\textwidth,height=\textheight]{../../Output/Figures/Slide_Fig_7_Combined_DWL.pdf}
\end{frame}

\begin{frame}{}
\phantomsection\label{section-7}
\BerkeleyMode

\begin{center}
\Huge \textbf{Structural Model [Simulated Data] }
\end{center}
\end{frame}

\begin{frame}{The Agent's Decision Problem}
\phantomsection\label{the-agents-decision-problem}
\BerkeleyLightMode

\SplitSlide{
    \textbf{State Space $x = (A, w, \rho)$}
    \begin{itemize}
        \item \textbf{Age ($A$):} 9 bins (0-5y ... 40y+).
        \item Risk $h(A)$ rises w/ age.
        \item \textbf{Wall ($w$):} Single vs. Double.
        \item \textbf{Regime ($\rho$):} Uniform Premium vs. Risk-Based.
    \end{itemize}
}{
    \textbf{Choice Set}
    \begin{enumerate}
        \item \textbf{Maintain ($d=m$)}
        \begin{itemize}
            \item Pay premium $p(x)$, face risk $h(x)$.
            \item Next: $A \to A+1$ (stochastic).
        \end{itemize}
        \item \textbf{Close ($d=c$)}
        \begin{itemize}
            \item Permeanlty Close Tank.
            \item Payoff: Scrap value $\kappa$.
        \end{itemize}
    \end{enumerate}
}

\vspace{0.5cm}

\textbf{Timing}

\begin{enumerate}
\tightlist
\item
  Facility observes state \(x_t\) and shock \(\varepsilon_{it}\) (Type I
  EV).
\item
  Chooses \(d_{it}\) to maximize expected discounted value
  (\(\beta = 0.95\)).
\end{enumerate}
\end{frame}

\begin{frame}{Bellman Equation \& Flow Utility}
\phantomsection\label{bellman-equation-flow-utility}
\BerkeleyLightMode

\textbf{Optimal Stopping Problem} \[
V(x) = \max\left\{ \underbrace{u^m(x) + \beta \mathbb{E}[V(x') | x, m]}_{\textbf{Maintain (Continuation)}}, \quad \underbrace{\Est{\kappa}}_{\textbf{Close (Scrap)}} \right\} + \sigma\varepsilon
\]

\textbf{Flow Utility (Normalized)} \[
u^m(x) = \underbrace{1}_{\textbf{Revenue}} - \underbrace{\Est{\gamma_{p}} \cdot p(x)}_{\textbf{Premium}} - \underbrace{\Est{\gamma_{r}} \cdot h(x) \cdot L}_{\textbf{Liability}}
\]

\begin{block}{Scaling \& Interpretation}
\phantomsection\label{scaling-interpretation}
\begin{itemize}
\tightlist
\item
  \textbf{Numeraire:} Annual net revenue is normalized to \(\psi = 1\).
\item
  \textbf{Unit Interpretation:} - Estimated parameters
  \(\{\Est{\kappa}, \Est{\gamma}\}\) are expressed in \textbf{multiples
  of annual revenue}.

  \begin{itemize}
  \tightlist
  \item
    A value of \(\Est{\gamma_r} = 0.5\) implies the marginal liability
    risk is equivalent to 6 months of revenue.
  \end{itemize}
\end{itemize}
\end{block}
\end{frame}

\begin{frame}{Identification of Parameters}
\phantomsection\label{identification-of-parameters}
\BerkeleyLightMode

\textbf{Which Variation Identifies Which Parameter?}

\begin{longtable}[]{@{}
  >{\centering\arraybackslash}p{(\columnwidth - 4\tabcolsep) * \real{0.3846}}
  >{\raggedright\arraybackslash}p{(\columnwidth - 4\tabcolsep) * \real{0.3077}}
  >{\raggedright\arraybackslash}p{(\columnwidth - 4\tabcolsep) * \real{0.3077}}@{}}
\toprule\noalign{}
\begin{minipage}[b]{\linewidth}\centering
Parameter
\end{minipage} & \begin{minipage}[b]{\linewidth}\raggedright
Interpretation
\end{minipage} & \begin{minipage}[b]{\linewidth}\raggedright
Source of Variation
\end{minipage} \\
\midrule\noalign{}
\endhead
\(\Est{\kappa}\) & Scrap Value & Average facility exit rate
(Unconditional Levels) \\
\(\Est{\gamma_{p}}\) & Premium Sensitivity & Cross-sectional premium
jumps across \((A, w, \rho)\) \\
\(\Est{\gamma_{r}}\) & Liability Sensitivity & Variation in deductible
sizes \& coverage limits \\
\bottomrule\noalign{}
\end{longtable}

\vspace{0.2cm}

\textbf{Estimation Protocol}

\begin{itemize}
\tightlist
\item
  \textbf{Algorithm:} Nested Pseudo-Likelihood (NPL)
  \footnotesize\{(Aguirregabiria \& Mira, 2007)\}
\item
  \textbf{Standard Errors:} Facility-level block bootstrap (\(B=999\)).
\item
  \textbf{Sample:} {[}Insert N{]} facilities observed quarterly
  (20XX-20XX).
\end{itemize}
\end{frame}

\begin{frame}{Estimation Results {[}Simulated Data for Demonstration
Only{]}}
\phantomsection\label{estimation-results-simulated-data-for-demonstration-only}
\BerkeleyLightMode

\begin{longtable}[]{@{}
  >{\raggedright\arraybackslash}p{(\columnwidth - 6\tabcolsep) * \real{0.2683}}
  >{\centering\arraybackslash}p{(\columnwidth - 6\tabcolsep) * \real{0.2439}}
  >{\centering\arraybackslash}p{(\columnwidth - 6\tabcolsep) * \real{0.0976}}
  >{\raggedright\arraybackslash}p{(\columnwidth - 6\tabcolsep) * \real{0.3902}}@{}}
\toprule\noalign{}
\begin{minipage}[b]{\linewidth}\raggedright
Parameter
\end{minipage} & \begin{minipage}[b]{\linewidth}\centering
Estimate
\end{minipage} & \begin{minipage}[b]{\linewidth}\centering
SE
\end{minipage} & \begin{minipage}[b]{\linewidth}\raggedright
Interpretation
\end{minipage} \\
\midrule\noalign{}
\endhead
\(\hat{\kappa}\) & 6.25 & (0.84) & Scrap value (annual revenue units) \\
\(\hat{\gamma}_{price}\) & −1.20 & (0.31) & Firms dislike premiums
(Price Elastic) \\
\(\hat{\gamma}_{risk}\) & 1.00 & (0.22) & Near-perfect private
internalization \\
\bottomrule\noalign{}
\end{longtable}

\vspace{0.5cm}

\textbf{Interpretation}

\begin{itemize}
\tightlist
\item
  \textbf{Price Sensitivity (\(\gamma < 0\))} Firms respond strongly to
  premium hikes (\(p < 0.01\)).
\item
  \textbf{Risk Neutrality (\(\gamma \approx 1\))} Owners fully
  internalize \emph{private} costs (deductibles), implying the remaining
  distortion is purely the \textbf{external} damage.
\end{itemize}
\end{frame}

\begin{frame}{Counterfactual Policy Scenarios}
\phantomsection\label{counterfactual-policy-scenarios}
\BerkeleyLightMode

\begin{longtable}[]{@{}
  >{\raggedright\arraybackslash}p{(\columnwidth - 4\tabcolsep) * \real{0.3333}}
  >{\raggedright\arraybackslash}p{(\columnwidth - 4\tabcolsep) * \real{0.3333}}
  >{\raggedright\arraybackslash}p{(\columnwidth - 4\tabcolsep) * \real{0.3333}}@{}}
\toprule\noalign{}
\begin{minipage}[b]{\linewidth}\raggedright
Scenario
\end{minipage} & \begin{minipage}[b]{\linewidth}\raggedright
Mechanism
\end{minipage} & \begin{minipage}[b]{\linewidth}\raggedright
Parameter Shift
\end{minipage} \\
\midrule\noalign{}
\endhead
\textbf{1. Baseline} & Status Quo (Estimated \(\hat{\theta}\)) &
\(\theta_{base} = \hat{\theta}\) \\
\textbf{2. Social Optimum} & Internalize Externality (Pigouvian Tax) &
\(\gamma_{risk}' = \gamma_{risk} \times \xi_{ext}\) \\
\textbf{3. Closure Subsidy} & Incentive Payment (Transfer) &
\(\kappa' = \kappa + \$10k\) \\
\textbf{4. Mandate} & Command \& Control (Targeted) & Force exit if
\(Age \geq 30\) \& SW \\
\bottomrule\noalign{}
\end{longtable}

\vspace{0.3cm}
\end{frame}

\begin{frame}{}
\phantomsection\label{section-8}
\LightMode

\includegraphics[width=0.95\textwidth,height=\textheight]{../../Output/Estimation_Results/Fleet_Survival.png}
\end{frame}

\begin{frame}{}
\phantomsection\label{section-9}
\LightMode

\includegraphics[width=0.95\textwidth,height=\textheight]{../../Output/Estimation_Results/Moral_Hazard.png}
\end{frame}

\begin{frame}{}
\phantomsection\label{section-10}
\LightMode

\includegraphics[width=0.95\textwidth,height=\textheight]{../../Output/Estimation_Results/Policy_Heterogeneity.png}
\end{frame}

\begin{frame}{}
\phantomsection\label{section-11}
\LightMode

\includegraphics[width=0.95\textwidth,height=\textheight]{../../Output/Estimation_Results/Welfare_Decomposition.png}
\end{frame}

\begin{frame}{Counterfactual Comparison}
\phantomsection\label{counterfactual-comparison}
\BerkeleyLightMode

\begin{table}
\caption{Counterfactual Welfare Analysis (Model B)} 
\centering
% \resizebox scales the tabular to the text width
\resizebox{\textwidth}{!}{
\begin{tabular}{lccccc}
  \toprule
Scenario & Closure Rate & Leak Risk & Firm Surplus & Social Welfare & $\Delta$ Welfare \\ 
  \midrule
Baseline & 5.1\% & 17.93\% & 31.96 & 26.76 & +0.000 \\ 
  Social Optimum Op & 37.3\% & 11.62\% & 29.98 & 26.67 & -0.088 \\ 
  Closure Subsidy & 7.8\% & 17.42\% & 31.54 & 26.48 & -0.274 \\ 
  Mandate & 13.4\% & 15.40\% & 31.40 & 27.23 & +0.467 \\ 
   \bottomrule
\end{tabular}
}
\label{tab:welfare_cf}
\end{table}
\end{frame}

\begin{frame}
\end{frame}

\end{document}
