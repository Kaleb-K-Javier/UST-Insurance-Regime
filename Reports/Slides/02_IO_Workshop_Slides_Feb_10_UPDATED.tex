\documentclass[aspectratio=169,]{beamer}

% -------------------------------------------------------------------------
% 1. REQUIRED PACKAGES (The "Batteries" Quarto usually includes)
% -------------------------------------------------------------------------
\usepackage{graphicx}
\usepackage{booktabs}
\usepackage{longtable}
\usepackage{hyperref}
\usepackage{array}      % <--- ADD THIS LINE HERE
\usepackage[most]{tcolorbox} % REQUIRED for Quarto Callouts
\usepackage{fontawesome5}    % REQUIRED for Callout Icons
\usepackage{framed}          % REQUIRED for standard blocks

% Fix for Pandoc's "tightlist"
\providecommand{\tightlist}{%
  \setlength{\itemsep}{0pt}\setlength{\parskip}{0pt}}

% -------------------------------------------------------------------------
% 2. CUSTOM DESIGN PACKAGES
% -------------------------------------------------------------------------
\usepackage{tikz}
\usepackage{calc}
\usepackage{xcolor}
\usepackage{xparse} 
\usepackage{etoolbox} 
\usepackage[default]{sourcesanspro} 
\usetikzlibrary{calc, positioning, shapes.geometric}

% -------------------------------------------------------------------------
% DEFINE QUARTO CALLOUT COLORS (REQUIRED)
% -------------------------------------------------------------------------
% We map these to your "Elegant Dark Mode" aesthetic.
% Note: Quarto expects these EXACT names.

% 1. Note (Blue)
\definecolor{quarto-callout-note-color}{HTML}{3498DB}        % Icon/Title Color
\definecolor{quarto-callout-note-color-frame}{HTML}{2980B9}  % Border Color

% 2. Tip (Green)
\definecolor{quarto-callout-tip-color}{HTML}{27AE60}
\definecolor{quarto-callout-tip-color-frame}{HTML}{2ECC71}

% 3. Important (Red/Crimson)
\definecolor{quarto-callout-important-color}{HTML}{C0392B}
\definecolor{quarto-callout-important-color-frame}{HTML}{E74C3C}

% 4. Warning (Orange)
\definecolor{quarto-callout-warning-color}{HTML}{F39C12}
\definecolor{quarto-callout-warning-color-frame}{HTML}{E67E22}

% 5. Caution (Yellow/Gold)
\definecolor{quarto-callout-caution-color}{HTML}{F1C40F}
\definecolor{quarto-callout-caution-color-frame}{HTML}{D4AC0D}

% -------------------------------------------------------------------------
% NEW: COLOR THEME COMMANDS (Robust Single-Slide Overrides)
% -------------------------------------------------------------------------

% 1. Light Mode Command
\newcommand{\LightMode}{
  % 1. Draw White Background (TikZ Overlay)
  \begin{tikzpicture}[remember picture, overlay]
    \fill[white] (current page.south west) rectangle (current page.north east);
  \end{tikzpicture}
  % 2. Force Text Colors to Black
  \setbeamercolor{normal text}{fg=black}
  \setbeamercolor{frametitle}{fg=black}
  \setbeamercolor{itemize item}{fg=black}
  \setbeamercolor{structure}{fg=black}
  \usebeamercolor[fg]{normal text}
}

% 2. Black Mode Command
\newcommand{\BlackMode}{
  \begin{tikzpicture}[remember picture, overlay]
    \fill[black] (current page.south west) rectangle (current page.north east);
  \end{tikzpicture}
  \setbeamercolor{normal text}{fg=white}
  \setbeamercolor{frametitle}{fg=white}
  \setbeamercolor{itemize item}{fg=white}
  \setbeamercolor{structure}{fg=white}
  \usebeamercolor[fg]{normal text}
}

% 3. Brand Mode Command
\newcommand{\BrandMode}{
  \begin{tikzpicture}[remember picture, overlay]
    \definecolor{tempBrand}{HTML}{7D181E}
    \fill[tempBrand] (current page.south west) rectangle (current page.north east);
  \end{tikzpicture}
  \setbeamercolor{normal text}{fg=white}
  \setbeamercolor{frametitle}{fg=white}
  \setbeamercolor{itemize item}{fg=white}
  \setbeamercolor{structure}{fg=white}
  \usebeamercolor[fg]{normal text}
}



% 4. UC Berkeley Theme (Modern Matte)
% A softer, less aggressive dark mode.
% - Uses a slightly desaturated blue to reduce eye strain.
% - Uses Gold ONLY for bullets/accents, not text.

% 4. UC Berkeley Theme (Modern Matte) - FIXED SCOPE
% 4. UC Berkeley Theme (Modern Matte) - FIXED WITH TITLE REDRAW
\newcommand{\BerkeleyMode}{
  % 1. DEFINE COLORS
  \definecolor{BerkeleyBlue}{HTML}{003262}
  \definecolor{CaliforniaGold}{HTML}{FDB515}
  \definecolor{BerkeleyMatte}{HTML}{0D2D52} 
  \definecolor{SoftWhite}{HTML}{F8F9FA}

  % 2. Draw Background AND Title
  \begin{tikzpicture}[remember picture, overlay]
    % Background Fill (Covers old title)
    \fill[BerkeleyMatte] (current page.south west) rectangle (current page.north east);
    % Gold Accent Strip
    \fill[CaliforniaGold] (current page.south west) rectangle ($(current page.north west)+(0.15cm,0)$);
    % REDRAW TITLE ON TOP
    \node[anchor=west, white, font=\Large\bfseries] at ($(current page.north west)+(1cm,-1cm)$) {\insertframetitle};
  \end{tikzpicture}
  
  % 3. Set Text Colors
  \setbeamercolor{normal text}{fg=SoftWhite}
  \setbeamercolor{frametitle}{fg=BerkeleyMatte} % Hide original title to prevent double-vision
  \setbeamercolor{itemize item}{fg=CaliforniaGold} 
  \setbeamercolor{structure}{fg=CaliforniaGold}
  
  \usebeamercolor[fg]{normal text}
}

% 5. UC Berkeley Theme (Academic Light) - FIXED WITH TITLE REDRAW
\newcommand{\BerkeleyLightMode}{
  % 1. DEFINE COLORS
  \definecolor{BerkeleyBlue}{HTML}{003262}
  \definecolor{CaliforniaGold}{HTML}{FDB515}
  
  % 2. Draw Background AND Title
  \begin{tikzpicture}[remember picture, overlay]
    % Background Fill
    \fill[white] (current page.south west) rectangle (current page.north east);
    % Blue Header Bar
    \fill[BerkeleyBlue] (current page.north west) rectangle ($(current page.north east)+(0,-1.4cm)$);
    % Gold Line
    \fill[CaliforniaGold] ($(current page.north west)+(0,-1.4cm)$) rectangle ($(current page.north east)+(0,-1.45cm)$);
    % REDRAW TITLE ON TOP
    \node[anchor=west, white, font=\Large\bfseries] at ($(current page.north west)+(1cm,-0.7cm)$) {\insertframetitle};
  \end{tikzpicture}
  
  % 3. Set Text Colors
  \setbeamercolor{normal text}{fg=black}
  \setbeamercolor{frametitle}{fg=white} 
  \setbeamercolor{itemize item}{fg=BerkeleyBlue}
  \setbeamercolor{structure}{fg=BerkeleyBlue}
  
  \usebeamercolor[fg]{normal text}
}


% -------------------------------------------------------------------------
% NEW: ACADEMIC POWER LAYOUTS
% -------------------------------------------------------------------------

% 1. Statement Layout (Big Impact Transitions)
% Usage: ::: {.layout-statement} Content :::
\newenvironment{layout-statement}{
  \vspace*{\fill}
  \centering
  \Huge \bfseries
  \setbeamercolor{normal text}{fg=CaliforniaGold} % Use Gold for impact
  \usebeamercolor[fg]{normal text}
}{
  \vspace*{\fill}
}

% 2. Dense Data Layout (Regression Tables)
% Usage: ::: {.layout-dense} Table :::
\newenvironment{layout-dense}{
  \vspace*{\fill}
  \tiny                  % Force tiny font
  \setlength\tabcolsep{2pt} % Tighten table columns
  \centering
}{
  \vspace*{\fill}
}

% 3. Split Slide Macro (Vertical Centering)
% Usage: \SplitSlide{Left Content}{Right Content}
\newcommand{\SplitSlide}[2]{
  \begin{columns}[c] % 'c' option forces vertical centering
    \begin{column}{0.48\textwidth}
      #1
    \end{column}
    \begin{column}{0.48\textwidth}
      \centering
      #2
    \end{column}
  \end{columns}
}

% -------------------------------------------------------------------------
% 3. DYNAMIC BACKGROUND & CONTRAST LOGIC
% -------------------------------------------------------------------------
  \definecolor{mainbg}{HTML}{0D2D52}

\setbeamercolor{background canvas}{bg=mainbg}

% Logic to calculate luminance and set text color automatically
\makeatletter
\newcommand{\AutoSetContrast}{
  \extractcolorspec{mainbg}{\mainbgspec}
  \expandafter\convertcolorspec\mainbgspec{rgb}\mainbgrgb
  \def\calculatebrightness##1,##2,##3;{
    \pgfmathparse{0.2126*##1 + 0.7152*##2 + 0.0722*##3}
  }
  \expandafter\calculatebrightness\mainbgrgb;
  \pgfmathparse{\pgfmathresult > 0.5 ? 1 : 0}
  \ifnum\pgfmathresult=1
    \definecolor{maintext}{HTML}{222222}
    \definecolor{dimtext}{HTML}{555555}
  \else
    \definecolor{maintext}{HTML}{FFFFFF}
    \definecolor{dimtext}{HTML}{CCCCCC}
  \fi
}
\makeatother

\AutoSetContrast

\setbeamercolor{normal text}{fg=maintext}
\setbeamercolor{frametitle}{fg=maintext}
\setbeamercolor{title}{fg=maintext}
\setbeamercolor{structure}{fg=maintext} 

% -------------------------------------------------------------------------
% 4. FOOTER CUSTOMIZATION
% -------------------------------------------------------------------------
\setbeamercolor{footline}{fg=dimtext}
\setbeamerfont{footline}{size=\tiny}

\newtoggle{hidesection}
\togglefalse{hidesection}

\setbeamertemplate{footline}{
  \begin{beamercolorbox}[wd=\paperwidth, ht=2.5ex, dp=1.125ex, leftskip=.3cm, rightskip=.3cm]{footline}
    \iftoggle{hidesection}{}{
      \insertsectionhead
    }
    \hfill
    \insertframenumber
  \end{beamercolorbox}
}
\setbeamertemplate{navigation symbols}{}

% -------------------------------------------------------------------------
% 5. CUSTOM COMMANDS & ENVIRONMENTS (FIXED FOR STABILITY)
% -------------------------------------------------------------------------
\NewDocumentCommand{\navbutton}{ O{} O{fill=gray} m }{
  \hyperlink{#1}{
    \tikz[baseline=(node.base)]{
      \node(node)[
        anchor=base,
        rounded corners=3pt,
        inner sep=5pt,
        text=white,
        font=\small\bfseries,
        #2
      ]{#3};
    }
  }
}

\newenvironment{layout-math}{
  \vspace*{\fill}
  \centering
  \begin{minipage}{\linewidth}
  \centering
}{
  \end{minipage}
  \vspace*{\fill}
}

% SAFE Layout for Tables (Removed risky \makebox)
\newenvironment{layout-table}{
  \vspace*{\fill}
  \begin{center}
}{
  \end{center}
  \vspace*{\fill}
}

% SAFE Layout for Full Figures (Uses LRBOX to capture content first)
\newsavebox{\fullfigbox}
\newenvironment{layout-full-figure}{
  \begin{lrbox}{\fullfigbox}
    \begin{minipage}{\paperwidth}
      \centering
}{
    \end{minipage}
  \end{lrbox}
  \begin{tikzpicture}[remember picture, overlay]
    \node[anchor=center, inner sep=0pt] at (current page.center) {
      \usebox{\fullfigbox}
    };
  \end{tikzpicture}
}

\newcommand{\hidefootersection}{\global\toggletrue{hidesection}}
\newcommand{\showfootersection}{\global\togglefalse{hidesection}}

% -------------------------------------------------------------------------
% 6. DOCUMENT BODY
% -------------------------------------------------------------------------
\title{Universal DiD Analysis: Texas UST Insurance Reform}
\author{Kaleb K.}

\begin{document}

\frame{\titlepage}

\begin{frame}{Motivation}
\phantomsection\label{motivation}
\BerkeleyMode

\textbf{Why Underground Storage Tanks Matter}

\begin{itemize}
\tightlist
\item
  580,000+ regulated USTs nationwide (EPA, 2023)
\item
  Primary source of groundwater contamination (EPA, 2023)
\item
  Median cleanup cost: \$250,000--\$500,000 per site
\item
  Long-tail risk: contamination discovered decades after installation
\end{itemize}

\textbf{The Policy Tension}

\begin{longtable}[]{@{}lll@{}}
\toprule\noalign{}
Regime & Premium Structure & Incentive Problem \\
\midrule\noalign{}
\endhead
Flat-Fee & \(p^F = \bar{p}\) & Moral hazard; no risk signal \\
Risk-Based & \(p^{RB}(a, w)\) & Price signal; may distort exit \\
\bottomrule\noalign{}
\end{longtable}
\end{frame}

\begin{frame}{Research Question \& Contribution}
\phantomsection\label{research-question-contribution}
\BerkeleyMode

\textbf{Research Question}

How does pricing environmental risk affect facility exit and technology
adoption decisions?

\vspace{0.5cm}

\textbf{Contributions}

\begin{enumerate}
\tightlist
\item
  \textbf{Data} Novel facility-year panel (1990--2023) linking Texas +
  18 control states
\item
  \textbf{Reduced Form} First causal evidence of risk-rating effects on
  UST behavior
\item
  \textbf{Structural} Dynamic model quantifying welfare gains
  (\(W_{RB} \gtrless W_{FF}\)?)
\end{enumerate}
\end{frame}

\begin{frame}{Preview of Findings}
\phantomsection\label{preview-of-findings}
\BerkeleyMode

\textbf{Reduced Form (DiD)}

\begin{itemize}
\tightlist
\item
  Risk-based pricing increased annual closure probability by
  \textasciitilde2--3 pp
\item
  Firms substituted toward \textbf{Retrofit} rather than pure Exit
\item
  Leak detection rates rose \textasciitilde1 pp (environmental benefit)
\end{itemize}

\textbf{Structural Dynamic Discrete Chocie Model }

\begin{itemize}
\tightlist
\item
  Price sensitivity: \(\hat{\gamma}_{price} < 0\) (firms respond to
  premiums)
\item
  Risk internalization: \(\hat{\gamma}_{risk} \approx 1\) (near-full
  private cost internalization)
\item
  Counterfactuals: Targeted subsidies outperform mandates
\end{itemize}
\end{frame}

\begin{frame}{Institutional Context: The 1999 Texas Reform}
\phantomsection\label{institutional-context-the-1999-texas-reform}
\BerkeleyMode

\textbf{Pre-1999: State-Run Flat-Fee Insurance}

\begin{itemize}
\tightlist
\item
  Uniform premium regardless of tank age/type
\item
  Low deductibles (\(D_F \approx 0.1L\))
\item
  Implicit cross-subsidy from safe to risky facilities
\end{itemize}

\textbf{Post-1999: Private Risk-Based Market}

\begin{itemize}
\tightlist
\item
  Premiums vary with age, wall type, and claims history
\item
  Higher deductibles (\(D_{RB} \approx 0.25L\))
\item
  Actuarially fair + loading:
  \(p^{RB}(a,w) = (1+\lambda) \cdot h(a,w) \cdot L\)
\end{itemize}

\textbf{Natural Experiment} Texas transitions; 18 control states retain
flat-fee
\end{frame}

\begin{frame}{Theoretical Framework: The ``Zombie Tank'' Problem}
\phantomsection\label{theoretical-framework-the-zombie-tank-problem}
\BerkeleyMode

\textbf{Firm's Dynamic Problem}

Each period, facility chooses:
\(d_t \in \{\text{Maintain}, \text{Close}\}\)

\textbf{Trade-off}

\begin{itemize}
\tightlist
\item
  \textbf{Maintain} Earn operating profit \(\pi(x_t)\), face leak risk
  \(h(a_t, w_t)\)
\item
  \textbf{Close} Receive scrap value \(\kappa\), exit market
\end{itemize}

\textbf{State Space} \(x = (A, w, \rho)\) where:

\begin{itemize}
\tightlist
\item
  \(A \in \{1,...,9\}\): Age bin (5-year intervals)
\item
  \(w \in \{\text{single}, \text{double}\}\): Wall type
\item
  \(\rho \in \{FF, RB\}\): Insurance regime
\end{itemize}
\end{frame}

\begin{frame}{Theoretical Framework: Optimal Stopping \& Insurance}
\phantomsection\label{theoretical-framework-optimal-stopping-insurance}
\BerkeleyMode

\textbf{The Firm's Dynamic Problem} A single facility maximizes expected
discounted value by choosing to \textbf{Maintain} or \textbf{Exit}.

\textbf{Flow Utility (Maintain)} \[
u(s) = \underbrace{R}_{\text{Net Revenue}} - \underbrace{C_j(s)}_{\text{Insurance Premium}} - \underbrace{h(s) \cdot L}_{\text{Exp. Leak Liability}}
\]

\begin{itemize}
\tightlist
\item
  \(R\) Annual revenue (less operating expenses)
\item
  \(L\) Private cost of a leak (Deductible + Cleanup)
\item
  \(h(s)\) Probability of a leak given state \(s\) (e.g., age)
\end{itemize}

\textbf{Decision Rule (Optimal Stopping)}

The firm exits when the liquidity of the land (scrap) exceeds the value
of operation

\[
\text{Exit if } \underbrace{V_{cont}(s)}_{\text{Continuation Value}} < \underbrace{\kappa}_{\text{Scrap Value}}
\]

\textbf{The Policy Lever Insurance Contracts (\(C_j\))} 1.
\textbf{Flat-Fee (\(F\))} \(C_F(s) = \bar{p}\)
\hfill (\textit{Risk Unpriced}) 2. \textbf{Risk-Based (\(RB\))}
\(C_{RB}(s) = (1+\lambda) \cdot h(s) \cdot L\)
\hfill (\textit{Risk Priced})
\end{frame}

\begin{frame}{Thought Experiment: The ``Zombie Tank''}
\phantomsection\label{thought-experiment-the-zombie-tank}
\BerkeleyMode

\textbf{Setup: A Representative Gas Station} We simulate a single
facility with fixed characteristics to isolate insurance incentives:

\begin{itemize}
\tightlist
\item
  \textbf{Revenue (\(R\))} \$100,000 / year (Net operating profit)
\item
  \textbf{Scrap Value (\(\kappa\))} \$400,000 (Land value net of closure
  costs)
\item
  \textbf{Leak Liability (\(L\))} \$200,000 (Private cleanup cost)
\item
  \textbf{Retrofit Cost (\(c_U\))} \$700,000 (Upgrade to Double-Wall)
\end{itemize}

\textbf{The Conflict} The firm faces rising leak risk \(h(a)\) as the
tank ages. The insurance contract determines how much of that risk is
priced into their annual bills.

\begin{itemize}
\tightlist
\item
  \textbf{Flat-Fee (\(F\))} Pays \textbf{\$2k/year} regardless of age.
  Deductible = \$20k.

  \begin{itemize}
  \tightlist
  \item
    \emph{Result:} Cost is fixed, creating an incentive to delay exit
    (``Zombie'').
  \end{itemize}
\item
  \textbf{Risk-Based (\(RB\))} Premium rises with risk (up to
  \textbf{\$10k+}). Deductible = \$50k.

  \begin{itemize}
  \tightlist
  \item
    \emph{Result:} Rising costs force efficient exit when
    \(V_{cont} < \$400k\).
  \end{itemize}
\end{itemize}
\end{frame}

\begin{frame}{The First Best: Optimal Closure}
\phantomsection\label{the-first-best-optimal-closure}
\BerkeleyMode

\includegraphics[width=0.95\textwidth,height=\textheight]{../../Output/Figures/Slide_Fig_2_SOC_Exit.pdf}
\end{frame}

\begin{frame}{3. Private Incentives: Risk-Based Pricing}
\phantomsection\label{private-incentives-risk-based-pricing}
\BerkeleyMode

\includegraphics[width=0.95\textwidth,height=\textheight]{../../Output/Figures/Slide_Fig_3_SOC_RB_Comparison.pdf}
\end{frame}

\begin{frame}{4. The ``Unavoidable'' Wedge (Risk-Based DWL)}
\phantomsection\label{the-unavoidable-wedge-risk-based-dwl}
\BerkeleyMode

\includegraphics[width=0.95\textwidth,height=\textheight]{../../Output/Figures/Slide_Fig_4_RB_DWL.pdf}
\end{frame}

\begin{frame}{5. Private Incentives: Flat-Fee Pricing}
\phantomsection\label{private-incentives-flat-fee-pricing}
\BerkeleyMode

\includegraphics[width=0.95\textwidth,height=\textheight]{../../Output/Figures/Slide_Fig_5_SOC_FF_Comparison.pdf}
\end{frame}

\begin{frame}{6. The ``Infinite Tail'' (Flat-Fee DWL)}
\phantomsection\label{the-infinite-tail-flat-fee-dwl}
\BerkeleyMode

\includegraphics[width=0.95\textwidth,height=\textheight]{../../Output/Figures/Slide_Fig_6_FF_Total_DWL.pdf}
\end{frame}

\begin{frame}{7. Summary: The Value of Reform}
\phantomsection\label{summary-the-value-of-reform}
\BerkeleyMode

\includegraphics[width=0.95\textwidth,height=\textheight]{../../Output/Figures/Slide_Fig_7_Combined_DWL.pdf}
\end{frame}

\begin{frame}{What About the Pollution Externality?}
\phantomsection\label{what-about-the-pollution-externality}
\BerkeleyMode

\textbf{Does Risk-Based Pricing Actually Reduce Pollution?}

The effect on observable LUST counts is theoretically ambiguous due to
two opposing forces:

\textbf{1. The ``Detection Effect'' (Increases Observable LUSTs)}
Private insurers mandate rigorous monitoring to minimize liability. -
\textbf{Enhanced Monitoring} More inspections \(\rightarrow\) higher
probability of detecting \emph{existing} leaks. - \textbf{Closure
Discovery} Tank removal requires site assessment \(\rightarrow\)
uncovers historical contamination. - \emph{Result:} Texas may report
\textbf{more} incidents initially, despite being safer.

\textbf{2. The ``Prevention Effect'' (Decreases True Risk)} Higher
premiums for risky tanks incentivize abatement. - \textbf{Early Exit}
High-risk tanks close before catastrophic failure. - \textbf{Retrofit}
Single-wall tanks are replaced with double-wall technology. -
\emph{Result:} The underlying stock of tanks becomes \textbf{safer} over
time.

\textbf{Net Result} Compositional shift toward lower true risk, even if
reported counts rise.
\end{frame}

\begin{frame}{From Theory to Empirics}
\phantomsection\label{from-theory-to-empirics}
\BerkeleyMode

\textbf{Testable Hypotheses from Toy Model}

\vspace{0.3cm}

\textbf{H0 (Age Profile)} Risk-based pricing skews age distribution \[
f_{RB}(A) < f_{FF}(A) \quad \text{for high-risk } A
\]

\textbf{H1 (Closure Hazard)} Risk-based pricing increases closure hazard
\[
h^{close}_{RB}(x) > h^{close}_{FF}(x) \quad \text{for high-risk } x
\]

\textbf{H2 (Selection)} High-risk types exit earlier under \(RB\) \[
\mathbb{E}[a | \text{exit}, RB] < \mathbb{E}[a | \text{exit}, FF]
\]

\textbf{H3 (Environment)} Aggregate leak rates decline \[
\mathbb{E}[\lambda | RB] < \mathbb{E}[\lambda | FF]
\]
\end{frame}

\begin{frame}{}
\phantomsection\label{section}
\BerkeleyMode

Reduced Form

Evidence
\end{frame}

\begin{frame}{Empirical Strategy: Data \& Sample}
\phantomsection\label{empirical-strategy-data-sample}
\BerkeleyLightMode

\textbf{Universe} EPA National UST Database + State Administrative
Records

\textbf{Sample Construction}

\begin{longtable}[]{@{}lll@{}}
\toprule\noalign{}
Filter & N Facilities & N Facility-Months \\
\midrule\noalign{}
\endhead
Raw Texas + 18 Controls & 297,533 & \textasciitilde60M \\
Active 1990--2023 & 185,000 & \textasciitilde45M \\
\textbf{Incumbent Filter} (Pre-1999) & 72,000 & \textasciitilde25M \\
\bottomrule\noalign{}
\end{longtable}

\vspace{0.3cm}

\begin{tcolorbox}[enhanced jigsaw, toptitle=1mm, leftrule=.75mm, opacityback=0, opacitybacktitle=0.6, colframe=quarto-callout-warning-color-frame, left=2mm, bottomrule=.15mm, toprule=.15mm, colback=white, bottomtitle=1mm, coltitle=black, breakable, title=\textcolor{quarto-callout-warning-color}{\faExclamationTriangle}\hspace{0.5em}{Warning}, titlerule=0mm, arc=.35mm, colbacktitle=quarto-callout-warning-color!10!white, rightrule=.15mm]

\textbf{Critical} DiD requires facilities active pre-1999 (no
post-reform entrants)

\end{tcolorbox}
\end{frame}

\begin{frame}{Identification Strategy}
\phantomsection\label{identification-strategy}
\BerkeleyLightMode

\textbf{Difference-in-Differences Specification}

\[
Y_{it} = \beta \cdot (\text{Texas}_i \times \text{Post}_t) + \alpha_i + \delta_t + X'_{it}\gamma + \varepsilon_{it}
\]

where:

\begin{itemize}
\tightlist
\item
  \(Y_{it} \in \{\text{LUST}, \text{Exit}, \text{Retrofit}\}\)
\item
  \(\alpha_i\): Facility fixed effects
\item
  \(\delta_t\): Year-month fixed effects
\item
  \(X_{it}\): Age bins, wall type, motor fuel indicator
\end{itemize}

\textbf{Clustering} State level (19 clusters); Webb WCB inference
\end{frame}

\begin{frame}{Results: Event Study (Leak Detection)}
\phantomsection\label{results-event-study-leak-detection}
\BerkeleyLightMode

\hidefootersection

\textbf{PLACE HOLDER FOR EVENT STUDY}

\showfootersection
\end{frame}

\begin{frame}{Results: Tank Closures (Extensive Margin)}
\phantomsection\label{results-tank-closures-extensive-margin}
\BerkeleyLightMode

\textbf{PLACE HOLDER FOR LUST STEP IN RESULTS}
\end{frame}

\begin{frame}{Mechanism: Exit vs.~Retrofit}
\phantomsection\label{mechanism-exit-vs.-retrofit}
\BerkeleyLightMode

\textbf{PLACE HOLDER FOR MECHANISM TABLE}

\textbf{Key Insight} Conditional on closure, Texas facilities chose
\textbf{Replacement} over \textbf{Exit}
\end{frame}

\begin{frame}{Reduced Form Summary}
\phantomsection\label{reduced-form-summary}
\BerkeleyLightMode

\textbf{DiD Findings}

\begin{longtable}[]{@{}lll@{}}
\toprule\noalign{}
Outcome & Treatment Effect & Interpretation \\
\midrule\noalign{}
\endhead
LUST Detection & XX.XX pp & More leaks discovered \\
Annual Closure & XX.XX pp & Accelerated exit \\
Retrofit Rate & XX.XX pp & Technology upgrading \\
\bottomrule\noalign{}
\end{longtable}

\vspace{0.5cm}

\textbf{Transition to Structural}

Reduced form identifies the \textbf{effect}, but we need a model for:

\begin{itemize}
\tightlist
\item
  \textbf{Welfare} Is \(W_{RB} > W_{FF}\)?
\item
  \textbf{Counterfactuals} What if subsidies? Mandates?
\end{itemize}
\end{frame}

\begin{frame}{}
\phantomsection\label{section-1}
\BerkeleyMode

Structural Model
\end{frame}

\begin{frame}{Dynamic Model Setup (Model B)}
\phantomsection\label{dynamic-model-setup-model-b}
\BerkeleyMode

\textbf{Binary Optimal Stopping Problem}

\[
V(x) = \max\left\{ \underbrace{u^m(x) + \beta \mathbb{E}[V(x') | x, d=m]}_{\text{Maintain}}, \quad \underbrace{\kappa}_{\text{Close}} \right\} + \sigma\varepsilon
\]

\textbf{Flow Utility (Maintain)}

\[
u^m(x) = \underbrace{\pi}_{\text{Revenue}} - \underbrace{\gamma_{price} \cdot p(A, w, \rho)}_{\text{Premium Cost}} - \underbrace{\gamma_{risk} \cdot h(A, w) \cdot L}_{\text{Expected Loss}}
\]

\textbf{Parameters to Estimate}
\(\theta = (\kappa, \gamma_{price}, \gamma_{risk})\)
\end{frame}

\begin{frame}{Identification of Parameters}
\phantomsection\label{identification-of-parameters}
\BerkeleyMode

\textbf{Parameter Recovery via NPL}

\begin{longtable}[]{@{}
  >{\raggedright\arraybackslash}p{(\columnwidth - 4\tabcolsep) * \real{0.2500}}
  >{\raggedright\arraybackslash}p{(\columnwidth - 4\tabcolsep) * \real{0.3409}}
  >{\raggedright\arraybackslash}p{(\columnwidth - 4\tabcolsep) * \real{0.4091}}@{}}
\toprule\noalign{}
\begin{minipage}[b]{\linewidth}\raggedright
Parameter
\end{minipage} & \begin{minipage}[b]{\linewidth}\raggedright
Identified By
\end{minipage} & \begin{minipage}[b]{\linewidth}\raggedright
Variation Source
\end{minipage} \\
\midrule\noalign{}
\endhead
\(\kappa\) (Scrap Value) & Closure hazard \emph{levels} &
Cross-sectional exit rates \\
\(\gamma_{price}\) (Price Sensitivity) & Response to premium schedule &
Age × Wall × Regime gradient \\
\(\gamma_{risk}\) (Risk Internalization) & Response to leak cost
exposure & Deductible variation \\
\bottomrule\noalign{}
\end{longtable}

\vspace{0.3cm}

\textbf{Estimation} Nested Pseudo-Likelihood (NPL) with K=2 iterations

\textbf{Standard Errors} Bootstrap (B=999) to account for policy
function estimation
\end{frame}

\begin{frame}{Estimation Results}
\phantomsection\label{estimation-results}
\BerkeleyLightMode
\end{frame}

\begin{frame}{Estimation Results}
\phantomsection\label{estimation-results-1}
\BerkeleyLightMode

\textbf{Model B Parameter Estimates}

\begin{longtable}[]{@{}
  >{\raggedright\arraybackslash}p{(\columnwidth - 6\tabcolsep) * \real{0.2683}}
  >{\centering\arraybackslash}p{(\columnwidth - 6\tabcolsep) * \real{0.2439}}
  >{\centering\arraybackslash}p{(\columnwidth - 6\tabcolsep) * \real{0.0976}}
  >{\raggedright\arraybackslash}p{(\columnwidth - 6\tabcolsep) * \real{0.3902}}@{}}
\toprule\noalign{}
\begin{minipage}[b]{\linewidth}\raggedright
Parameter
\end{minipage} & \begin{minipage}[b]{\linewidth}\centering
Estimate
\end{minipage} & \begin{minipage}[b]{\linewidth}\centering
SE
\end{minipage} & \begin{minipage}[b]{\linewidth}\raggedright
Interpretation
\end{minipage} \\
\midrule\noalign{}
\endhead
\(\hat{\kappa}\) & 6.25 & (0.84) & Scrap value (annual revenue units) \\
\(\hat{\gamma}_{price}\) & −1.20 & (0.31) & Firms dislike premiums
(Price Elastic) \\
\(\hat{\gamma}_{risk}\) & 1.00 & (0.22) & Near-perfect private
internalization \\
\bottomrule\noalign{}
\end{longtable}

\vspace{0.5cm}

\textbf{Interpretation}

\begin{itemize}
\tightlist
\item
  \textbf{Price Sensitivity (\(\gamma < 0\))} Firms respond strongly to
  premium hikes (\(p < 0.01\)).
\item
  \textbf{Risk Neutrality (\(\gamma \approx 1\))} Owners fully
  internalize \emph{private} costs (deductibles), implying the remaining
  distortion is purely the \textbf{external} damage.
\end{itemize}
\end{frame}

\begin{frame}{Estimated Value Function}
\phantomsection\label{estimated-value-function}
\BerkeleyMode

\textbf{Firm's Continuation Value \(V(x)\)}

\includegraphics[width=0.95\textwidth,height=\textheight]{../../Output/Estimation_Results/CF_Value_Function.png}
\end{frame}

\begin{frame}{Counterfactual Design}
\phantomsection\label{counterfactual-design}
\BerkeleyLightMode

\textbf{Four Policy Scenarios}

\begin{longtable}[]{@{}
  >{\raggedright\arraybackslash}p{(\columnwidth - 4\tabcolsep) * \real{0.1034}}
  >{\raggedright\arraybackslash}p{(\columnwidth - 4\tabcolsep) * \real{0.3448}}
  >{\raggedright\arraybackslash}p{(\columnwidth - 4\tabcolsep) * \real{0.5517}}@{}}
\toprule\noalign{}
\begin{minipage}[b]{\linewidth}\raggedright
\#
\end{minipage} & \begin{minipage}[b]{\linewidth}\raggedright
Scenario
\end{minipage} & \begin{minipage}[b]{\linewidth}\raggedright
Implementation
\end{minipage} \\
\midrule\noalign{}
\endhead
1 & \textbf{Baseline} & Status quo (TX: RB, Controls: FF) \\
2 & \textbf{Social Optimum} & \(\gamma_{risk} \times 2\) (internalize
externality) \\
3 & \textbf{Closure Subsidy} & \(\kappa + \text{Subsidy}\) (pay to
close) \\
4 & \textbf{Mandate} & Force closure if \(A \geq 30\) \&
Single-Walled \\
\bottomrule\noalign{}
\end{longtable}

\vspace{0.3cm}

\textbf{Welfare Metrics}

\begin{itemize}
\tightlist
\item
  Average closure probability
\item
  Expected leak risk: \(\mathbb{E}[h(x) \cdot P(\text{maintain}|x)]\)
\item
  Social loss: Private loss \(\times\) externality multiplier
\end{itemize}
\end{frame}

\begin{frame}{Counterfactual Results: Closure by Policy}
\phantomsection\label{counterfactual-results-closure-by-policy}
\BerkeleyLightMode

\textbf{PLACE HOLDER FOR COUNTERFACTUAL FIGURE}
\end{frame}

\begin{frame}{Counterfactual Comparison}
\phantomsection\label{counterfactual-comparison}
\BerkeleyLightMode

\textbf{Welfare Summary}

\begin{longtable}[]{@{}
  >{\raggedright\arraybackslash}p{(\columnwidth - 8\tabcolsep) * \real{0.1351}}
  >{\raggedright\arraybackslash}p{(\columnwidth - 8\tabcolsep) * \real{0.2162}}
  >{\raggedright\arraybackslash}p{(\columnwidth - 8\tabcolsep) * \real{0.2973}}
  >{\raggedright\arraybackslash}p{(\columnwidth - 8\tabcolsep) * \real{0.1486}}
  >{\raggedright\arraybackslash}p{(\columnwidth - 8\tabcolsep) * \real{0.2027}}@{}}
\toprule\noalign{}
\begin{minipage}[b]{\linewidth}\raggedright
Scenario
\end{minipage} & \begin{minipage}[b]{\linewidth}\raggedright
Avg Close Rate
\end{minipage} & \begin{minipage}[b]{\linewidth}\raggedright
\(\Delta\) vs Baseline
\end{minipage} & \begin{minipage}[b]{\linewidth}\raggedright
Leak Risk
\end{minipage} & \begin{minipage}[b]{\linewidth}\raggedright
\(\Delta\) Risk
\end{minipage} \\
\midrule\noalign{}
\endhead
Baseline & 4.2\% & --- & 1.8\% & --- \\
Social Opt & 8.1\% & +3.9 pp & 0.9\% & −50\% \\
Subsidy & 6.5\% & +2.3 pp & 1.2\% & −33\% \\
Mandate & 5.8\% & +1.6 pp & 1.4\% & −22\% \\
\bottomrule\noalign{}
\end{longtable}

\vspace{0.3cm}

\textbf{Ranking} Social Optimum \(>\) Subsidy \(>\) Mandate \(>\)
Baseline
\end{frame}

\begin{frame}{Conclusion \& Policy Implications}
\phantomsection\label{conclusion-policy-implications}
\BerkeleyMode

\textbf{Findings}

\begin{enumerate}
\tightlist
\item
  \textbf{Risk-based pricing works} +2--3 pp closure rates (DiD)
\item
  \textbf{Mechanism is upgrading, not abandonment} Retrofit substitution
\item
  \textbf{Welfare-improving but not First Best} Gap to Social Optimum
  remains
\end{enumerate}

\textbf{Policy Recommendations}

\begin{itemize}
\tightlist
\item
  Risk-based insurance dominates flat-fee pooling
\item
  Targeted \textbf{closure subsidies} for high-risk tanks can close
  remaining welfare gap
\item
  \textbf{Mandates} are less efficient (distort margins not requiring
  intervention)
\end{itemize}

\textbf{Broader Implication} Market-based environmental regulation can
outperform command-and-control when behavioral elasticities are
sufficient.
\end{frame}

\end{document}
